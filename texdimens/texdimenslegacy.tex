% \texdimenUUup, \texdimenUUdown, the 0.9gamma versions
% -----------------------------------------------------
% archiving the up and down macros from 0.9gamma release
% (also to check if new up and down macros are backwards
%  compatible and do tests to compare the efficiencies)
%
% The 0.9gamma algorithm was based on
%
% R = round((T+0.5)/phi) if 1<phi<2, round(T/phi) if phi>2
%
% which gives the \texdimenUU output.
%
% This was kept in 1.0, but "up/down" macros were re-done:
%
% Formerly, finding "up=Zu" and "down=Zd" was done by checking
% if trunc(R phi) (as computed via \dimexpr tricks) proved
% at least equal to T (thus Zu=R) or < T (then Zu=R+1) and
% whether similarly it was
% at most equal to T (then Zd=R) or > T (then Zd=R-1)
%
% This got superseded at 1.0 by direct computation of Zd and
% Zu via their ceil()-based formulae (see code comments
% in texdimens.tex)
%
% Note: I kept in this file definitions of the \texdimenUU's
% but all tests will do \input texdimenslegacy.tex first,
% to be certain they use the stuff from \input texdimens.tex
% whenever it has definition in both.
\edef\texdimenslegacyendinput{\endlinechar\the\endlinechar%
\catcode`\noexpand _=\the\catcode`\_%
\catcode`\noexpand @=\the\catcode`\@\relax\noexpand\endinput}%
\endlinechar13\relax%
\catcode`\_=11 \catcode`\@=11 % only for using \p@ (also \z@ now) of Plain. Check exists?
%
\def\texdimenfirstofone#1{#1}%
{\catcode`p 12\catcode`t 12
 \csname expandafter\endcsname\gdef\csname texdimenstrippt\endcsname#1pt{#1}}%
%
% down macros:
% for units with phi < 2:
\def\texdimendown_A#1{\if-#1\texdimendown_neg\fi\texdimendown_B#1}%
\def\texdimendown_B#1;#2;{\expandafter\texdimendown_c\the\numexpr(2*#1+1)#2;#1;}%
% for units with phi > 2:
\def\texdimendown_a#1{\if-#1\texdimendown_neg\fi\texdimendown_b#1}%
\def\texdimendown_b#1;#2;{\expandafter\texdimendown_c\the\numexpr#1#2;#1;}%
% shared macros:
\def\texdimendown_c#1;{\expandafter\texdimendown_d\the\dimexpr#1sp;#1;}%
{\catcode`P 12\catcode`T 12\lowercase{\gdef\texdimendown_d#1PT};#2;#3;#4;%
   {\ifdim#1#4>#3sp \texdimendown_e{#2}\fi\texdimenfirstofone{#1}}%
}%
% this #2 will be \fi
\def\texdimendown_e#1#2#3#4{#2\expandafter\texdimenstrippt\the\dimexpr\numexpr#1-1sp\relax}%
% negative branch:
% The problem here is that if input very small, output can be 0.0, and we
% do not want -0.0 as output.
% So let's do this somewhat brutally and non-efficiently.
% Anyhow, negative inputs are not our priority.
% #1 is \fi here and #2 is \texdimendown_b or _B:
\def\texdimendown_neg#1#2-#3;#4;#5;{#1\expandafter\texdimenstrippt\the\dimexpr-#2#3;#4;#5;pt\relax}%
%
% up macros:
\def\texdimenup_A#1{\if-#1\texdimenup_neg\fi\texdimenup_B#1}%
\def\texdimenup_B#1;#2;{\expandafter\texdimenup_c\the\numexpr(2*#1+1)#2;#1;}%
\def\texdimenup_a#1{\if-#1\texdimenup_neg\fi\texdimenup_b#1}%
\def\texdimenup_b#1;#2;{\expandafter\texdimenup_c\the\numexpr#1#2;#1;}%
\def\texdimenup_c#1;{\expandafter\texdimenup_d\the\dimexpr#1sp;#1;}%
{\catcode`P 12\catcode`T 12\lowercase{\gdef\texdimenup_d#1PT};#2;#3;#4;%
   {\ifdim#1#4<#3sp \texdimenup_e{#2}\fi\texdimenfirstofone{#1}}%
}%
% this #2 will be \fi
\def\texdimenup_e#1#2#3#4{#2\expandafter\texdimenstrippt\the\dimexpr\numexpr#1+1sp\relax}%
% negative branch:
% Here we can me more expeditive than for the "down" macros.
% But this breaks f-expandability.
% #1 will be \fi and #2 is \texdimenup_b or _B:
\def\texdimenup_neg#1#2-{#1-#2}%
%
% pt
%
\def\texdimenpt#1{\expandafter\texdimenstrippt\the\dimexpr#1\relax}%
%
% bp 7227/7200 = 803/800
%
\def\texdimenbp#1{\expandafter\texdimenbp_\the\numexpr\dimexpr#1;}%
\def\texdimenbp_#1#2;{%
    \expandafter\texdimenstrippt\the\dimexpr\numexpr(2*#1#2+\if-#1-\fi1)*400/803sp\relax
}%
% \texdimenbpdown: maximal dim exactly expressible in bp and at most equal to input
\def\texdimenbpdownlegacy#1{\expandafter\texdimendown_A\the\numexpr\dimexpr#1;*400/803;bp;}%
% \texdimenbpup: minimal dim exactly expressible in bp and at least equal to input
\def\texdimenbpuplegacy#1{\expandafter\texdimenup_A\the\numexpr\dimexpr#1;*400/803;bp;}%
%
% nd 685/642
%
\def\texdimennd#1{\expandafter\texdimennd_\the\numexpr\dimexpr#1;}%
\def\texdimennd_#1#2;{%
    \expandafter\texdimenstrippt\the\dimexpr\numexpr(2*#1#2+\if-#1-\fi1)*321/685sp\relax
}%
% \texdimennddown: maximal dim exactly expressible in nd and at most equal to input
\def\texdimennddownlegacy#1{\expandafter\texdimendown_A\the\numexpr\dimexpr#1;*321/685;nd;}%
% \texdimenndup: minimal dim exactly expressible in nd and at least equal to input
\def\texdimennduplegacy#1{\expandafter\texdimenup_A\the\numexpr\dimexpr#1;*321/685;nd;}%
%
% dd 1238/1157
%
\def\texdimendd#1{\expandafter\texdimendd_\the\numexpr\dimexpr#1;}%
\def\texdimendd_#1#2;{%
    \expandafter\texdimenstrippt\the\dimexpr\numexpr(2*#1#2+\if-#1-\fi1)*1157/2476sp\relax
}%
% \texdimendddown: maximal dim exactly expressible in dd and at most equal to input
\def\texdimendddownlegacy#1{\expandafter\texdimendown_A\the\numexpr\dimexpr#1;*1157/2476;dd;}%
% \texdimenddup: minimal dim exactly expressible in dd and at least equal to input
\def\texdimendduplegacy#1{\expandafter\texdimenup_A\the\numexpr\dimexpr#1;*1157/2476;dd;}%
%
% mm 7227/2540 phi now >2, use from here on the simpler approach
%
\def\texdimenmm#1{\expandafter\texdimenstrippt\the\dimexpr(#1)*2540/7227\relax}%
% \texdimenmmdown: maximal dim exactly expressible in mm and at most equal to input
\def\texdimenmmdownlegacy#1{\expandafter\texdimendown_a\the\numexpr\dimexpr#1;*2540/7227;mm;}%
% \texdimenmmup: minimal dim exactly expressible in mm and at least equal to input
\def\texdimenmmuplegacy#1{\expandafter\texdimenup_a\the\numexpr\dimexpr#1;*2540/7227;mm;}%
%
% pc 12/1
%
\def\texdimenpc#1{\expandafter\texdimenstrippt\the\dimexpr(#1)/12\relax}%
% \texdimenpcdown: maximal dim exactly expressible in pc and at most equal to input
\def\texdimenpcdownlegacy#1{\expandafter\texdimendown_a\the\numexpr\dimexpr#1;/12;pc;}%
% \texdimenpcup: minimal dim exactly expressible in pc and at least equal to input
\def\texdimenpcuplegacy#1{\expandafter\texdimenup_a\the\numexpr\dimexpr#1;/12;pc;}%
%
% nc 1370/107
%
\def\texdimennc#1{\expandafter\texdimenstrippt\the\dimexpr(#1)*107/1370\relax}%
% \texdimenncdown: maximal dim exactly expressible in nc and at most equal to input
\def\texdimenncdownlegacy#1{\expandafter\texdimendown_a\the\numexpr\dimexpr#1;*107/1370;nc;}%
% \texdimenncup: minimal dim exactly expressible in nc and at least equal to input
\def\texdimenncuplegacy#1{\expandafter\texdimenup_a\the\numexpr\dimexpr#1;*107/1370;nc;}%
%
% cc 14856/1157
%
\def\texdimencc#1{\expandafter\texdimenstrippt\the\dimexpr(#1)*1157/14856\relax}%
% \texdimenccdown: maximal dim exactly expressible in cc and at most equal to input
\def\texdimenccdownlegacy#1{\expandafter\texdimendown_a\the\numexpr\dimexpr#1;*1157/14856;cc;}%
% \texdimenccup: minimal dim exactly expressible in cc and at least equal to input
\def\texdimenccuplegacy#1{\expandafter\texdimenup_a\the\numexpr\dimexpr#1;*1157/14856;cc;}%
%
% cm 7227/254
%
\def\texdimencm#1{\expandafter\texdimenstrippt\the\dimexpr(#1)*254/7227\relax}%
% \texdimencmdown: maximal dim exactly expressible in cm and at most equal to input
\def\texdimencmdownlegacy#1{\expandafter\texdimendown_a\the\numexpr\dimexpr#1;*254/7227;cm;}%
% \texdimencmup: minimal dim exactly expressible in cm and at least equal to input
\def\texdimencmuplegacy#1{\expandafter\texdimenup_a\the\numexpr\dimexpr#1;*254/7227;cm;}%
%
% in 7227/100
%
\def\texdimenin#1{\expandafter\texdimenstrippt\the\dimexpr(#1)*100/7227\relax}%
% \texdimenindown: maximal dim exactly expressible in in and at most equal to input
\def\texdimenindownlegacy#1{\expandafter\texdimendown_a\the\numexpr\dimexpr#1;*100/7227;in;}%
% \texdimeninup: minimal dim exactly expressible in in and at least equal to input
\def\texdimeninuplegacy#1{\expandafter\texdimenup_a\the\numexpr\dimexpr#1;*100/7227;in;}%
%
%
% \texdimenwithunit, the 1.0 version
% ----------------------------------
\def\texdimenwithunitlegacy#1#2{\expandafter\texdimenwithunitlegacy_i
% no premultiplication of dim1 by 2 as was done for technical
% reasons when dim2<1pt branch used round((2T+1)*32768/f)
    \the\numexpr\dimexpr#2\expandafter;\the\numexpr\dimexpr#1;}%
%
% so tempted to do \input xintkernel.sty to have some utilities...
% not even a \@gobble in Plain...
\let\texdimenorthat\texdimenfirstofone
\def\texdimendothis#1#2\texdimenorthat#3{\fi#1}%
\def\texdimengobtominus#1-{}%
\def\texdimenwithunitlegacy_i#1{%
     \texdimengobtominus#1\texdimenwithunitlegacy_switchsigns-%
     \texdimenwithunitlegacy_j#1%
}%
\def\texdimenwithunitlegacy_switchsigns-\texdimenwithunitlegacy_j-#1;#2%
{%
% due to \texdimenwithunitlegacy_Bneg we can not simply prefix dim1
% with -, as -0 is bad there. So let's check also if #2 is 0
    \texdimenzerominusfork
      #2-\texdimenwithunitlegacy_Bzero % abusive double usage
      0#2\texdimenwithunitlegacy_j     % abusive shortcut
       0-{\texdimenwithunitlegacy_ic#2}%
    \krof
    #1;%
}%
\def\texdimenwithunitlegacy_ic#1#2;{\texdimenwithunitlegacy_j#2;-#1}%
\def\texdimenwithunitlegacy_j#1;#2{%
        % \ifnum#1=\p@\texdimendothis\texdimenwithunitlegacy_p@\fi
        \ifnum#1>\p@\texdimendothis\texdimenwithunitlegacy_A\fi
        \texdimenorthat\texdimenwithunitlegacy_B#2#1;%
}%
% not needed, as the "ceil" branch is fine to use for f = 65536
% \def\texdimenwithunitlegacy_p@#1#2;#3;{%
%     \expandafter\texdimenstrippt\the\dimexpr#1#3sp/2\relax
% }%
% unit>1pt, handle this as for bp. Attention it would be wrong for
% unit=1pt!
\def\texdimenwithunitlegacy_A#1#2;#3;{%
    \expandafter\texdimenstrippt
    \the\dimexpr\numexpr(2*#1#3+\if-#1-\fi1)*32768/#2sp\relax
    % - fine if dim1>0, <0, or =0
    % - with *\p@ better but an early doubled dim2 would complicate 1pt
    % test and not sure if doing \p@/(2*#2) here advantageous
}%
% unit<=1pt.
% if dim1<0, simply negate result for dim1>0 as it can not possibly be 0.0
% Indeed T*65536/f will be at least 1 so its ceil also (in fact ceil
% will even be at least 2 if f<65536).
% The dim1=0 case must get filtered out due to way of calculating the
% "ceil" in \numexpr
\def\texdimenzerominusfork #10-#2#3\krof {#2}%
\def\texdimenwithunitlegacy_B#1{\texdimenzerominusfork
                           #1-\texdimenwithunitlegacy_Bzero
                           0#1\texdimenwithunitlegacy_Bneg
                           0-\texdimenwithunitlegacy_Ba
                          \krof#1}%
\def\texdimenwithunitlegacy_Ba#1#2;#3;{%
    % no overflow possible from 2*#1#3=2*dim1
    \expandafter\texdimenwithunitlegacy_Bb\the\numexpr(2*#1#3-#2)/(2*#2);#1#3;#2;%
}%
% now k;T;f;. Get the remainder r=T-k*f
\def\texdimenwithunitlegacy_Bb#1;#2;#3;{%
    \expandafter\texdimenwithunitlegacy_Bc\the\numexpr#2-#1*#3;#1;#3;%
}%
% now r;k;f;. We can start \the\numexpr k+ ....
% and there we will need to get R=round(r*65536/f), Ept=Rsp,
% check if E"f sp"<"r sp", if yes replace R by R+1 else keep E etc.
\def\texdimenwithunitlegacy_Bc#1;#2;#3;{%
% \the\numexpr k+0.ddddd is handy because it can well be actually
% \the\numexpr k+1.0, now that we use ceil approach in this branch
    \the\numexpr#2+\expandafter\texdimenwithunitlegacy_Bd
                   \the\numexpr #1*\p@/#3;#1;#3;%
}%
% R;r;f;
\def\texdimenwithunitlegacy_Bd#1;{%
    \expandafter\texdimenwithunitlegacy_Be\the\dimexpr#1sp;#1;%
}%
% Ept;R;r;f;
% #1=0.ddd... or 1.0 but has no end marker hence the
% \texdimenfirstofone{#1} as in \texdimendown_d and \texdimenup_d
{\catcode`P 12\catcode`T 12
\lowercase{\gdef\texdimenwithunitlegacy_Be#1PT};#2;#3;#4;{%
    \ifdim#1\dimexpr#4sp<#3sp \texdimenwithunitlegacy_Bf{#2}\fi
    \texdimenfirstofone{#1}%
    }%
}%
% #2 is \fi. Add a dimen storage \onesp for 1sp?
\def\texdimenwithunitlegacy_Bf#1#2\texdimenfirstofone#3{#2%
    \expandafter\texdimenstrippt\the\dimexpr#1sp+1sp\relax
}%
% Here definitely not caring about f-expandability. Or efficiency.
\def\texdimenwithunitlegacy_Bneg-{-\texdimenwithunitlegacy_Ba{}}%
\def\texdimenwithunitlegacy_Bzero#1;#2;{0.0}%
\texdimenslegacyendinput
%! Local variables:
%! mode: TeX
%! fill-column: 1000
%! sentence-end-double-space: t
%! End:
