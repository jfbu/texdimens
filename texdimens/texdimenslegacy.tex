% archival of the up and down macros from 0.9gamma release
% they were supersed at 1.0 by macros whose principle
% is decribed here in the "ceil()" section
\edef\texdimenslegacyendinput{\endlinechar\the\endlinechar%
\catcode`\noexpand _=\the\catcode`\_%
\catcode`\noexpand @=\the\catcode`\@\relax\noexpand\endinput}%
\endlinechar13\relax%
\catcode`\_=11 \catcode`\@=11 % only for using \p@ (also \z@ now) of Plain. Check exists?
%
% Mathematics ("down" and "up" macros)
% ===========
%
% In the entire discussion here, "uu" stands for some core unit,
% or some unit corresponding to a dimension > 1pt. For the case
% of a unit corresponding to a dimension < 1pt, i.e. to
% \texdimenwithunit macro added at 0.99, refer to the
% comments of issue #2 on the tracker site.
%
% Is T sp attainable from unit "uu"?.
% If not, what is largest dimension < Tsp which is attainable?
% Here we suppose T>0.
%
% phi>1, psi=1/phi, psi<1.
%
%     U(N,phi)=trunc(N phi) is the strictly increasing sequence,
%     indexed by non-negative integers, of attainable dimensions.
%     (in sp unit)
%
%     U(N)<= T <  U(N+1)    iff    N = ceil((T+1)psi) - 1
%     U(M)<  T <= U(M+1)    iff    M = ceil(T psi)    - 1
%
% In other words:
%
% - the largest attainable dimension not exceeding T sp
%   is obtained via the integer "Zd = ceil((T+1)psi) - 1 = N",
%   (i.e. find D with Zd=round(65536 D) then "D uu" is "down" approximation)
%
% - the smallest attainable dimension at least equal to T sp
%   is obtained from the integer "Zu = ceil(T psi) = M + 1"
%
% - the two "Z"'s are either equal (i.e. T is attained) or Zu=Zd+1.
%
% Stumbling block
% ---------------
%
% I will in what follows refer to trunc(), floor() or ceil() only for
% positive arguments, obtained as ratios x/y or sometimes as a numexpr
% "scaling" operation" x*y/z which uses temporarily use doubled
% precision.
%
% As \numexpr's x/y is round(x/y), with rounding away from zero, we have
% access to floor(t) for t>=0 as round(t+0.5)-1 and for t>0 also as
% round(t-0.5). The former may cause overflow as it involves
% (2x+y)/(2y) but the latter (2x-y)/(2y) will not overflow if x comes
% from a dimension as 2x<2**31 then.
%
% ceil(t) is more complex as it is floor(t)+1 only for t not an integer.
% Below I will explain some tricks, but roughly the a priori prejudice
% is that computing ceil() is costly. And this is not only a prejudice,
% it is an actual fact.
%
% This package initial core idea was to base the analysis and
% implementation on an alternative quantity to the Zu = ceil(T psi) and
% Zd = ceil((T+1)psi) - 1, is R = round((T+0.5)psi) because it is
% much more accessible to direct computation in \numexpr. (notice in
% particular that 2*T+1<2**31 will never cause overflow).

% This documentation referred to the ceil(t) based formula as "stumbling
% block" because a naive handling, for example **assuming t is not an
% integer**, would mean looking at round(t+0.5). For example for the "in"
% unit with psi = 100/7227 this leads to computation such as
% round((T*200+7227)/14454). This already raises the problem that 200*T
% will easily overflow. Computation of R = round((T+0.5)psi) however
% had no such issue at all, so the implementation became naturally based
% upon it.
%
% And the computation of Zu and Zd was based upon getting R first, as is
% described in the next section, not on their ceil() based formulae.
%
% For the units "dd", "nc", "in", this approach had a limitation that
% \maxdimen or very close dimensions could not be used as inputs to
% the up and down macros. At 1.0, an alternative approach based
% on the ceil() formulae for Zd and Zu was implemented. Here is
% its detailed escription added 2021/11/07 in the case of Zd/"in".
%
% We want Zd = ceil((T+1)*100/7227) - 1, with T assumed positive.
%
% Let T = k*7227 + r with 0<= r < 7227, 0<=k, and r>0 if k=0.
%
% (T+1)*100/7227 becomes 100*k + (r+1)*100/7227 and thus
%
% Zd = 100 * k + ceil(x) - 1
%
% with  x = n*100/7227, and n = 1+r, so 0<n<=7227
%
% Here we have a nice situation 0 < x <= 100. Then:
% 
% ceil(x) = 100 - floor(100 - x)
%         = 100 - (round(100 - x + 0.5) - 1)
%         = 101 - round(100 * (1 - n/7227) + 0.5)
%         = 101 - round((200 * (7227 - n) + 7227)/14454)
%
% We can thus achieve the computation of Zd = ceil((T+1)*100/7227) - 1
% for T>0 without overflow in \numexpr this way:
%
%     k = floor(T/7227) = round(T/7227 - 0.5)
%                       = round((2*T - 7227) / 14454)  (T>0 used here)
%
%     r = T - 7227 * k  = T modulo 7227
%
%     Zd = 100 * k + 100 - round( (201*7227 - 200*(r+1))/14454 )
%
% Everything here is computable within \numexpr and has absolutely
% no potential overflow problem at all. However it has number of steps
% and computations.
%
% The same analysis can be done for Zu = ceil(T*100/7227) and for all
% core TeX units.
%
% The round((T+0.5)*psi) based approach
% --------------------------------------
%
% Recall in all of this T > 0. And phi>1, psi=1/phi<1.
%
% Let's return to our analysis of the
%  U(N)<= T < U(N+1) and U(M)< T <= U(M+1) equations.
%
% We will also use the N=Zd, and M+1=Zu notations of the introduction.
%
% case1:  M = N, i.e. Zd<Zu, i.e. T is not attainable:
%         M=N=Zd < T psi < (T+1) psi <= N+1=Zu
%
%         Then clearly R = round((T+0.5)psi) is either Zd or Zu.
%         We will not know which one before computing trunc(R phi)
%         and check if it is < T or > T.
%
%         As will be explained later trunc(R phi) can be computed very
%         easily by hijacking TeX's handling of dimensions, we don't
%         have to launch into \numexpr evaluations for that.
%
% case2:  M = N - 1, i.e. T = Zd = Zu is attained:
%         T psi <= N < (T+1) psi, T = trunc(N phi)
%
%         Let v=(T+0.5)psi. As v = T psi + 0.5 psi it is < N+0.5
%         And as v = (T+1)psi - 0.5psi it is > N - 0.5.
%         So R = round(v) = N.
%
% We thus have the initial observation which was at the coe of this
% package initial release:
%
% - compute R = round((T+0.5) psi)
% - if T is attained, then T = trunc(R * phi)
% - if T is not attained then Zd = R and Zu = R+1 or Zd = R-1 and Zu =
% R.
%
% How do we check if R = Zd or Zu? We need to evaluate trunc(R phi) and
% compare it with T. But this trunc(R phi) can be computed the following
% way:
%
% - obtain D pt from \the\dimexpr R sp. Knuth's algorithm guarantees
%   that R = round(D * 65536)
% - then D uu where uu is the unit with conversion factor phi is
%   converted by TeX into "trunc(R phi) sp", i.e. 
%   trunc(R phi) = \number\dimexpr Duu\relax, where D pt = \the\dimexpr Rsp\relax.
%
% Conclusion:
% 1. the macro \texdimenuu does the one-liner R=round((T+0.5) psi)
%    then \the\dimexpr Rsp\relax and strips the "pt" unit
% 2. macros \texdimenuuup and \texdimenuudown go further and check
%    which one of Zd or Zu is R, obtaining thus Zd or Zu.
%
% For units with conversion factor phi>2, a simplification is possible.
% In that case let X = round(T psi) (it has the advantage compared to 
% R that we can apply the formula without checking the sign of T).
%
% Going back to our earlier analyis, now with psi < 0.5 (1uu>2pt)
%
% case1: T is not attainable
%        M=N=Zd < T psi < (T+1) psi <= N+1=Zu 
%        As Zd < T psi < Zu, we have round(T psi) = Zd or Zu
%
% case2: T is attained, i.e. T psi <= N < (T+1) psi.
%        As psi<0.5, and T psi + psi > N, we have T psi > N - 0.5.
%        And T psi <= N so N = round(T psi).
%
% So, for psi < 0.5, the X=round(T psi) can play the same role as
% R=round((T+0.5)psi). If T is attained, we get the decimal D from this
% X and if T is not attained we know that X is either Zd or Zu.
%
% The computations of X and Y=trunc(X phi) can be done independently of
% sign of T.  But the final test has to be changed to Y < T if T < 0 and
% then one must replace X by X+1. So we must filter out the sign of the
% input.
%
% Going back to the 1<phi<2 case, psi>0.5, then it would be slightly
% less costly to compute X = round(T psi) than R = round((T + 0.5) psi),
% but if we then realize that trunc(X phi) < T we do not yet know if
% trunc((X+1) phi) = T or is > T, i.e. we don't know if Zd =X or X+1,
% and we can not tell yet if T is attained or not.
%
% In contrast if we find out that trunc(R phi)<T, we then know for sure
% that Zd=R, Zu=R+1 and that T is not attained.
%
% Problems with \maxdimen and the obtention of Zu and Zd
% ------------------------------------------------------
%
% Obtaining R = round((T+0.5)psi) has no risk of overflow.
% But checking as described above which one of Zd or Zu (or both)
% is R goes via a test computation which will cause overflow
% if by bad luck R = Zu and Zu will give rise to a decimal D
% such that D uu > \maxdimen.
%
% For T=\maxdimen (or very close) this is what happens for the units
% "dd", "nc", and "in".
%
% And using X=round(T psi) for the "in" which has phi>2 in place of R
% does not help.
% Choosing the R or X approaches has no decisive impact on the potential
% "Dimension too large error" problem. 
%
% So I chose to use the round(T psi) approach for all units with phi>2
% and the round((T+0.5) phi) approach for the units with 1<phi<2.
%
% In future, the \texdimenUU{up,down} for UU = dd, nc, in, will probably
% be re-implemented using the alternatvie approach to Zd and Zu
% described in the section added 2021/11/07 on the ceil() function.
% This approach will allow the macros to accept \maxdimen as input.
% 
%
% Implementation
% ==============
%
\def\texdimenfirstofone#1{#1}%
{\catcode`p 12\catcode`t 12
 \csname expandafter\endcsname\gdef\csname texdimenstrippt\endcsname#1pt{#1}}%
%
% down macros:
% for units with phi < 2:
\def\texdimendown_A#1{\if-#1\texdimendown_neg\fi\texdimendown_B#1}%
\def\texdimendown_B#1;#2;{\expandafter\texdimendown_c\the\numexpr(2*#1+1)#2;#1;}%
% for units with phi > 2:
\def\texdimendown_a#1{\if-#1\texdimendown_neg\fi\texdimendown_b#1}%
\def\texdimendown_b#1;#2;{\expandafter\texdimendown_c\the\numexpr#1#2;#1;}%
% shared macros:
\def\texdimendown_c#1;{\expandafter\texdimendown_d\the\dimexpr#1sp;#1;}%
{\catcode`P 12\catcode`T 12\lowercase{\gdef\texdimendown_d#1PT};#2;#3;#4;%
   {\ifdim#1#4>#3sp \texdimendown_e{#2}\fi\texdimenfirstofone{#1}}%
}%
% this #2 will be \fi
\def\texdimendown_e#1#2#3#4{#2\expandafter\texdimenstrippt\the\dimexpr\numexpr#1-1sp\relax}%
% negative branch:
% The problem here is that if input very small, output can be 0.0, and we
% do not want -0.0 as output.
% So let's do this somewhat brutally and non-efficiently.
% Anyhow, negative inputs are not our priority.
% #1 is \fi here and #2 is \texdimendown_b or _B:
\def\texdimendown_neg#1#2-#3;#4;#5;{#1\expandafter\texdimenstrippt\the\dimexpr-#2#3;#4;#5;pt\relax}%
%
% up macros:
\def\texdimenup_A#1{\if-#1\texdimenup_neg\fi\texdimenup_B#1}%
\def\texdimenup_B#1;#2;{\expandafter\texdimenup_c\the\numexpr(2*#1+1)#2;#1;}%
\def\texdimenup_a#1{\if-#1\texdimenup_neg\fi\texdimenup_b#1}%
\def\texdimenup_b#1;#2;{\expandafter\texdimenup_c\the\numexpr#1#2;#1;}%
\def\texdimenup_c#1;{\expandafter\texdimenup_d\the\dimexpr#1sp;#1;}%
{\catcode`P 12\catcode`T 12\lowercase{\gdef\texdimenup_d#1PT};#2;#3;#4;%
   {\ifdim#1#4<#3sp \texdimenup_e{#2}\fi\texdimenfirstofone{#1}}%
}%
% this #2 will be \fi
\def\texdimenup_e#1#2#3#4{#2\expandafter\texdimenstrippt\the\dimexpr\numexpr#1+1sp\relax}%
% negative branch:
% Here we can me more expeditive than for the "down" macros.
% But this breaks f-expandability.
% #1 will be \fi and #2 is \texdimenup_b or _B:
\def\texdimenup_neg#1#2-{#1-#2}%
%
% pt
%
\def\texdimenpt#1{\expandafter\texdimenstrippt\the\dimexpr#1\relax}%
%
% bp 7227/7200 = 803/800
%
\def\texdimenbp#1{\expandafter\texdimenbp_\the\numexpr\dimexpr#1;}%
\def\texdimenbp_#1#2;{%
    \expandafter\texdimenstrippt\the\dimexpr\numexpr(2*#1#2+\if-#1-\fi1)*400/803sp\relax
}%
% \texdimenbpdown: maximal dim exactly expressible in bp and at most equal to input
\def\texdimenbpdownlegacy#1{\expandafter\texdimendown_A\the\numexpr\dimexpr#1;*400/803;bp;}%
% \texdimenbpup: minimal dim exactly expressible in bp and at least equal to input
\def\texdimenbpuplegacy#1{\expandafter\texdimenup_A\the\numexpr\dimexpr#1;*400/803;bp;}%
%
% nd 685/642
%
\def\texdimennd#1{\expandafter\texdimennd_\the\numexpr\dimexpr#1;}%
\def\texdimennd_#1#2;{%
    \expandafter\texdimenstrippt\the\dimexpr\numexpr(2*#1#2+\if-#1-\fi1)*321/685sp\relax
}%
% \texdimennddown: maximal dim exactly expressible in nd and at most equal to input
\def\texdimennddownlegacy#1{\expandafter\texdimendown_A\the\numexpr\dimexpr#1;*321/685;nd;}%
% \texdimenndup: minimal dim exactly expressible in nd and at least equal to input
\def\texdimennduplegacy#1{\expandafter\texdimenup_A\the\numexpr\dimexpr#1;*321/685;nd;}%
%
% dd 1238/1157
%
\def\texdimendd#1{\expandafter\texdimendd_\the\numexpr\dimexpr#1;}%
\def\texdimendd_#1#2;{%
    \expandafter\texdimenstrippt\the\dimexpr\numexpr(2*#1#2+\if-#1-\fi1)*1157/2476sp\relax
}%
% \texdimendddown: maximal dim exactly expressible in dd and at most equal to input
\def\texdimendddownlegacy#1{\expandafter\texdimendown_A\the\numexpr\dimexpr#1;*1157/2476;dd;}%
% \texdimenddup: minimal dim exactly expressible in dd and at least equal to input
\def\texdimendduplegacy#1{\expandafter\texdimenup_A\the\numexpr\dimexpr#1;*1157/2476;dd;}%
%
% mm 7227/2540 phi now >2, use from here on the simpler approach
%
\def\texdimenmm#1{\expandafter\texdimenstrippt\the\dimexpr(#1)*2540/7227\relax}%
% \texdimenmmdown: maximal dim exactly expressible in mm and at most equal to input
\def\texdimenmmdownlegacy#1{\expandafter\texdimendown_a\the\numexpr\dimexpr#1;*2540/7227;mm;}%
% \texdimenmmup: minimal dim exactly expressible in mm and at least equal to input
\def\texdimenmmuplegacy#1{\expandafter\texdimenup_a\the\numexpr\dimexpr#1;*2540/7227;mm;}%
%
% pc 12/1
%
\def\texdimenpc#1{\expandafter\texdimenstrippt\the\dimexpr(#1)/12\relax}%
% \texdimenpcdown: maximal dim exactly expressible in pc and at most equal to input
\def\texdimenpcdownlegacy#1{\expandafter\texdimendown_a\the\numexpr\dimexpr#1;/12;pc;}%
% \texdimenpcup: minimal dim exactly expressible in pc and at least equal to input
\def\texdimenpcuplegacy#1{\expandafter\texdimenup_a\the\numexpr\dimexpr#1;/12;pc;}%
%
% nc 1370/107
%
\def\texdimennc#1{\expandafter\texdimenstrippt\the\dimexpr(#1)*107/1370\relax}%
% \texdimenncdown: maximal dim exactly expressible in nc and at most equal to input
\def\texdimenncdownlegacy#1{\expandafter\texdimendown_a\the\numexpr\dimexpr#1;*107/1370;nc;}%
% \texdimenncup: minimal dim exactly expressible in nc and at least equal to input
\def\texdimenncuplegacy#1{\expandafter\texdimenup_a\the\numexpr\dimexpr#1;*107/1370;nc;}%
%
% cc 14856/1157
%
\def\texdimencc#1{\expandafter\texdimenstrippt\the\dimexpr(#1)*1157/14856\relax}%
% \texdimenccdown: maximal dim exactly expressible in cc and at most equal to input
\def\texdimenccdownlegacy#1{\expandafter\texdimendown_a\the\numexpr\dimexpr#1;*1157/14856;cc;}%
% \texdimenccup: minimal dim exactly expressible in cc and at least equal to input
\def\texdimenccuplegacy#1{\expandafter\texdimenup_a\the\numexpr\dimexpr#1;*1157/14856;cc;}%
%
% cm 7227/254
%
\def\texdimencm#1{\expandafter\texdimenstrippt\the\dimexpr(#1)*254/7227\relax}%
% \texdimencmdown: maximal dim exactly expressible in cm and at most equal to input
\def\texdimencmdownlegacy#1{\expandafter\texdimendown_a\the\numexpr\dimexpr#1;*254/7227;cm;}%
% \texdimencmup: minimal dim exactly expressible in cm and at least equal to input
\def\texdimencmuplegacy#1{\expandafter\texdimenup_a\the\numexpr\dimexpr#1;*254/7227;cm;}%
%
% in 7227/100
%
\def\texdimenin#1{\expandafter\texdimenstrippt\the\dimexpr(#1)*100/7227\relax}%
% \texdimenindown: maximal dim exactly expressible in in and at most equal to input
\def\texdimenindownlegacy#1{\expandafter\texdimendown_a\the\numexpr\dimexpr#1;*100/7227;in;}%
% \texdimeninup: minimal dim exactly expressible in in and at least equal to input
\def\texdimeninuplegacy#1{\expandafter\texdimenup_a\the\numexpr\dimexpr#1;*100/7227;in;}%
%
\texdimenslegacyendinput
%! Local variables:
%! mode: TeX
%! fill-column: 1000
%! sentence-end-double-space: t
%! End:
