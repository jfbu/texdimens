\input texdimenslegacy
\input texdimens

\input xintkernel.sty

\def\test#1#2#3#4#5{%
\expandafter\let\expandafter\NEW\csname texdimen#4#5\endcsname
% turns out the old versions are NOT quicker...
\expandafter\let\expandafter\OLD\csname texdimen#4#5legacy\endcsname
\pdfresettimer
\xintReplicate{#1}{\xintReplicate{#2}{\edef\x{\NEW{#3}}}}%
\edef\T{\expandafter\texdimenstrippt\the\dimexpr\pdfelapsedtime sp}%
\immediate\write128{\T s (ceil, \the\numexpr#1*#2\relax\space reps)}%
\pdfresettimer
\xintReplicate{#1}{\xintReplicate{#2}{\edef\y{\OLD{#3}}}}%
\edef\U{\expandafter\texdimenstrippt\the\dimexpr\pdfelapsedtime sp}%
\immediate\write128{\U s (round, \the\numexpr#1*#2\relax\space reps)}%
\immediate\write128{\U/\T\space about \texdimenwithunit{\U pt}{\T pt} for #4#5
with #3}%
\ifx\x\y\else\ERROR\fi
\immediate\write128{}%
}

\test{100}{1000}{1000000000sp}{in}{up}

\test{100}{1000}{1000000000sp}{in}{down}

\test{100}{1000}{1000000000sp}{nc}{up}

\test{100}{1000}{1000000000sp}{nc}{down}

\test{100}{1000}{1000000000sp}{dd}{up}

\test{100}{1000}{1000000000sp}{dd}{down}

%
\test{100}{1000}{1234567sp}{in}{up}

\test{100}{1000}{1234567sp}{in}{down}

\test{100}{1000}{1234567sp}{nc}{up}

\test{100}{1000}{1234567sp}{nc}{down}

\test{100}{1000}{1234567sp}{dd}{up}

\test{100}{1000}{1234567sp}{dd}{down}

%
\test{100}{1000}{70000sp}{in}{up}

\test{100}{1000}{70000sp}{in}{down}

\test{100}{1000}{70000sp}{nc}{up}

\test{100}{1000}{70000sp}{nc}{down}

\test{100}{1000}{70000sp}{dd}{up}

\test{100}{1000}{70000sp}{dd}{down}

%
\test{100}{1000}{57481sp}{in}{up}

\test{100}{1000}{57481sp}{in}{down}

\test{100}{1000}{57481sp}{nc}{up}

\test{100}{1000}{57481sp}{nc}{down}

\test{100}{1000}{57481sp}{dd}{up}

\test{100}{1000}{57481sp}{dd}{down}

%
\test{100}{1000}{9999sp}{in}{up}

\test{100}{1000}{9999sp}{in}{down}

\test{100}{1000}{9999sp}{nc}{up}

\test{100}{1000}{9999sp}{nc}{down}

\test{100}{1000}{9999sp}{dd}{up}

\test{100}{1000}{9999sp}{dd}{down}


\bye

It looks as if, barring bad luck in my random choices of inputs,
the new «ceil» based macros are definitly usually faster than
the «round» based ones. Probably because of the latter
overhead of the dim comparison which may requires a second run.
Possibly the few exceptions are when this second run does not
happen.
