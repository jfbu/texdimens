% etex '\def\RSEED{<randomseed}\input benchmark_allupdown_random.tex'
% (or the \RSEED will itself be randomly generated)
% Conclusion:
% - the old macros are faster for some inputs
% - but these inputs seem relatively rare
% - and the difference is never much more than 2% or 3%
% - on the other hand, often the new macros are 5%-10%
%   faster, sometimes even 20%-30%
% Also, new output is always exactly like old in all testing done
% (but this is no proof, I had at some point earlier a typo in ncup
%  which was never caught by testing; but since I generated all
%  new macros out of a template)
\input texdimenslegacy
\input texdimens

\input xintkernel.sty

\input xinttools.sty

\ifdefined\RSEED\else\edef\RSEED{\pdfuniformdeviate 100000000}\fi

\newwrite\out
\immediate\openout\out=\jobname-\RSEED.txt
\immediate\write\out{Random seed: \RSEED}
\immediate\write128{Random seed: \RSEED}

\pdfsetrandomseed \RSEED\relax

\def\test#1#2#3#4{%
\edef\w{\pdfuniformdeviate 131072}%
\expandafter\let\expandafter\NEW\csname texdimen#3#4\endcsname
\expandafter\let\expandafter\OLD\csname texdimen#3#4legacy\endcsname
%
    \pdfresettimer
    \xintReplicate{#1}{\xintReplicate{#2}{\edef\x{\NEW{\w sp}}}}%
    \edef\T{\expandafter\texdimenstrippt\the\dimexpr\pdfelapsedtime sp}%
\immediate\write128{\T s (ceil, \the\numexpr#1*#2\relax\space reps)}%
\immediate\write\out{\T s (ceil, \the\numexpr#1*#2\relax\space reps)}%
%
    \pdfresettimer
    \xintReplicate{#1}{\xintReplicate{#2}{\edef\y{\OLD{\w sp}}}}%
    \edef\U{\expandafter\texdimenstrippt\the\dimexpr\pdfelapsedtime sp}%
\immediate\write128{\U s (round, \the\numexpr#1*#2\relax\space reps)}%
\immediate\write\out{\U s (round, \the\numexpr#1*#2\relax\space reps)}%
%
\immediate\write128{\U/\T = about \texdimenwithunit{\U pt}{\T pt} for #3#4 with \w sp}%
\immediate\write\out{\U/\T= about \texdimenwithunit{\U pt}{\T pt} for #3#4 with \w sp}%
\ifx\x\y\else\ERROR\fi
\immediate\write128{}%
\immediate\write\out{}%
}

\def\Test#1#2#3#4{\test{#1}{#2}{#3}{#4}%
                  \test{#1}{#2}{#3}{#4}%
                  \test{#1}{#2}{#3}{#4}}

\def\TEST{%
\xintFor ##1 in {%
bp,
nd,
dd,
mm,
pc,
nc,
cc,
cm,
in
}:{\xintFor ##2 in {up, down}:{\Test{100}{1000}{##1}{##2}}%
    }%
}%

\TEST

%\TEST

%\TEST

\bye
