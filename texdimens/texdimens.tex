% This is file texdimens.tex, part of texdimens package, which
% is distributed under the LPPL 1.3c. Copyright (c) 2021 Jean-François B.
% 2021/11/10 v1.0
\edef\texdimensendinput{\endlinechar\the\endlinechar%
\catcode`\noexpand _=\the\catcode`\_%
\catcode`\noexpand @=\the\catcode`\@\relax\noexpand\endinput}%
\endlinechar13\relax%
\catcode`\_=11 \catcode`\@=11 % only for using \p@ (also \z@ now) of Plain. Check exists?
% so tempted to do \input xintkernel.sty to have some utilities...
% not even a \@gobble in Plain...
\def\texdimengobtominus#1-{}%
\def\texdimenzerominusfork #10-#2#3\krof {#2}%
%
% Mathematics (the challenge of the "down" and "up" macros)
% ===========
%
% In the entire discussion here, "uu" stands for some core unit,
% or some unit corresponding to an internal dimension > 1pt.
%
% Is T sp attainable from unit "uu"?.
% If not, what is largest dimension < Tsp which is attainable?
% Here we suppose T>0.
%
% D uu --> N = round(D * 65536) --> T = trunc (N * phi)
%
% phi>1 is the conversion factor associated to "uu"
% psi=1/phi, psi<1. Define U(N, phi) = trunc (N * phi)
% 
%     U(N,phi) is thus the strictly increasing sequence,
%     indexed by non-negative integers, of non-negative
%     attainable dimensions. (in sp unit)
%
% T>0, then:
%
%     U(N)<= T <  U(N+1)    iff    N = ceil((T+1)psi) - 1
%     U(M)<  T <= U(M+1)    iff    M = ceil(T psi)    - 1
%
% In other words:
%
% - the largest attainable dimension not exceeding T sp
%   is obtained via the integer "Zd = ceil((T+1)psi) - 1 = N",
%   (i.e. find D with Zd=round(65536 D) then "D uu" is "down" approximation)
%
% - the smallest attainable dimension at least equal to T sp
%   is obtained from the integer "Zu = ceil(T psi) = M + 1"
%
% - the two "Z"'s are either equal (i.e. T is attained) or Zu=Zd+1.
%
% The round((T+0.5)*psi) based approach (basis of the "\texdimenUU" macros)
% -------------------------------------
%
% Recall in all of this T > 0. And phi>1, psi=1/phi<1.
%
% Let's return to our analysis of the
%
%         U(N)<= T < U(N+1) and U(M)< T <= U(M+1)
%
% equations. We will also use the N=Zd, and M+1=Zu notations.
%
% case1:  M = N, i.e. Zd<Zu, i.e. T is not attainable:
%         M=N=Zd < T psi < (T+1) psi <= N+1=Zu
%
%         Then clearly R = round((T+0.5)psi) is either Zd or Zu.
%         We will not know which one before computing trunc(R phi)
%         and check if it is < T or > T.
%
%         As will be explained later trunc(R phi) can be computed very
%         easily by hijacking TeX's handling of dimensions, we don't
%         have to launch into \numexpr evaluations for that.
%
% case2:  M = N - 1, i.e. T = Zd = Zu is attained:
%         T psi <= N < (T+1) psi, T = trunc(N phi)
%
%         Let v=(T+0.5)psi. As v = T psi + 0.5 psi it is < N+0.5
%         And as v = (T+1)psi - 0.5psi it is > N - 0.5.
%         So R = round(v) = N.
%
% We thus have the initial observation which was at the core of this
% package initial release:
%
% - compute R = round((T+0.5) psi)
%
% - if T is attained, then T = trunc(R * phi)
%
% - if T is not attained then either { Zd = R and Zu = R+1 }  or 
%    {Zd = R-1 and Zu = R}.
%
%   How do we check if R = Zd or Zu? We need to evaluate trunc(R phi) and
%   compare it with T. This trunc(R phi) can be computed the following way:
%
%   - obtain D pt from \the\dimexpr R sp. Knuth's algorithm guarantees
%   that R = round(D * 65536)
%
%   - then D uu where uu is the unit with conversion factor phi is
%     converted by TeX into "trunc(R phi) sp", i.e.  trunc(R phi) =
%     \number\dimexpr Duu\relax, where D pt = \the\dimexpr Rsp\relax.
%
% Conclusion:
%
% 1. the macro \texdimenuu does the one-liner R=round((T+0.5) psi)
%    then \the\dimexpr Rsp\relax and strips the "pt" unit
%
% 2. macros \texdimenuuup and \texdimenuudown go further and check
%    which one of Zd or Zu is R, obtaining thus Zd or Zu.
%    [update: this describes the macros prior to 1.0 release]
%
% For units with conversion factor phi>2, a simplification is possible.
% In that case let X = round(T psi) (it has the advantage compared to 
% R that we can apply the formula without checking the sign of T).
%
% Going back to our earlier analyis, now with psi < 0.5 (1uu>2pt)
%
% case1: T is not attainable
%        M=N=Zd < T psi < (T+1) psi <= N+1=Zu 
%        As Zd < T psi < Zu, we have round(T psi) = Zd or Zu
%
% case2: T is attained, i.e. T psi <= N < (T+1) psi.
%        As psi<0.5, and T psi + psi > N, we have T psi > N - 0.5.
%        And T psi <= N so N = round(T psi).
%
% So, for psi < 0.5, the X=round(T psi) can play the same role as
% R=round((T+0.5)psi). If T is attained, we get the decimal D from this
% X and if T is not attained we know that X is either Zd or Zu.
%
% The computations of X and Y=trunc(X phi) can be done independently of
% sign of T.  But the final test has to be changed to Y < T if T < 0 and
% then one must replace X by X+1. So we must filter out the sign of the
% input.
%
% Going back to the 1<phi<2 case, psi>0.5, then it would be slightly
% less costly to compute X = round(T psi) than R = round((T + 0.5) psi),
% but if we then realize that trunc(X phi) < T we do not yet know if
% trunc((X+1) phi) = T or is > T, i.e. we don't know if Zd =X or X+1,
% and we can not tell yet if T is attained or not.
%
% In contrast if we find out that trunc(R phi)<T, we then know for sure
% that Zd=R, Zu=R+1 and that T is not attained.
%
% Problems with \maxdimen in the obtention of Zu and Zd
% -----------------------------------------------------
%
% Obtaining R = round((T+0.5)psi) has no risk of overflow.
% But checking as described above which one of Zd or Zu (or both)
% is R goes via a test computation which will cause overflow
% if by bad luck R = Zu and Zu will give rise to a decimal D
% such that D uu > \maxdimen.
%
% For T=\maxdimen (or very close) this is what happens for the units
% "dd", "nc", and "in".
%
% Besides, it turns out that this test which is done to decide whether
% R=Zu or R=Zd, and on which the initial implementation of the macros
% "up" and "down" was done at 0.9 gamma release is a bit costly.
%
% At 1.0 release, all the "up" and "down" macros were re-implemented
% via a more stubborn usage of the ceil() based formulae for Zd and Zu.
% This made all usable even with \maxdimen input and besided, proved
% on average slightly faster.
%
% Overcoming the ceil() stumbling block (i.e. the 1.0 manner for "up/down")
% ------------------------------------- 
%
% I will in what follows refer to trunc(), floor() or ceil() only for
% positive arguments, obtained as ratios x/y or sometimes as a numexpr
% "scaling" operation" x*y/z which uses temporarily use doubled
% precision.
%
% As \numexpr's x/y is round(x/y), with rounding away from zero, we have
% access to floor(t) for t>=0 as round(t+0.5)-1 and for t>0 also as
% round(t-0.5). The former may cause overflow as it involves
% (2x+y)/(2y) but the latter (2x-y)/(2y) will not overflow if x comes
% from a dimension as 2x<2**31 then.
%
% ceil(t) is more complex as it is floor(t)+1 only for t not an integer.
% Let's explain how to overcome the challenge for Zd and the "in" unit,
% i.e. a conversion factor of 7227/100.
%
% We want Zd = ceil((T+1)*100/7227) - 1, with T assumed positive.
%
% Let T = k*7227 + r with 0<= r < 7227, 0<=k, and r>0 if k=0.
%
% (T+1)*100/7227 becomes 100*k + (r+1)*100/7227 and thus
%
% Zd = 100 * k + ceil(x) - 1
%
% with  x = n*100/7227, and n = 1+r, so 0<n<=7227
%
% Here we have a nice situation 0 < x <= 100. Then:
% 
% ceil(x) = 100 - floor(100 - x)
%         = 100 - (round(100 - x + 0.5) - 1)
%         = 101 - round(100 * (1 - n/7227) + 0.5)
%         = 101 - round((200 * (7227 - n) + 7227)/14454)
%
% We can thus achieve the computation of Zd = ceil((T+1)*100/7227) - 1
% for T>0 without overflow in \numexpr this way:
%
%     k = floor(T/7227) = round(T/7227 - 0.5)
%                       = round((2*T - 7227) / 14454)  (T>0 used here)
%
%     r = T - 7227 * k  = T modulo 7227
%
%     Zd = 100 * k + 100 - round( (201*7227 - 200*(r+1))/14454 )
%
% Everything here is computable within \numexpr and has absolutely no
% potential overflow problem at all. The same analysis can be done for
% Zu = ceil(T*100/7227) and for all core TeX units. See the comments
% below for all obtained formulae and some additional details.
%
% Implementation
% ==============
%
\def\texdimenfirstofone#1{#1}%
{\catcode`p 12\catcode`t 12
 \csname expandafter\endcsname\gdef\csname texdimenstrippt\endcsname#1pt{#1}}%
%
% pt
%
\def\texdimenpt#1{\expandafter\texdimenstrippt\the\dimexpr#1\relax}%
%
% bp 7227/7200 = 803/800
%
\def\texdimenbp#1{\expandafter\texdimenbp_\the\numexpr\dimexpr#1;}%
\def\texdimenbp_#1#2;{%
    \expandafter\texdimenstrippt\the\dimexpr\numexpr(2*#1#2+\if-#1-\fi1)*400/803sp\relax
}%
%
% nd 685/642
%
\def\texdimennd#1{\expandafter\texdimennd_\the\numexpr\dimexpr#1;}%
\def\texdimennd_#1#2;{%
    \expandafter\texdimenstrippt\the\dimexpr\numexpr(2*#1#2+\if-#1-\fi1)*321/685sp\relax
}%
%
% dd 1238/1157
%
\def\texdimendd#1{\expandafter\texdimendd_\the\numexpr\dimexpr#1;}%
\def\texdimendd_#1#2;{%
    \expandafter\texdimenstrippt\the\dimexpr\numexpr(2*#1#2+\if-#1-\fi1)*1157/2476sp\relax
}%
%
% mm 7227/2540 phi now >2, use from here on the X = round(T psi) approach
%
\def\texdimenmm#1{\expandafter\texdimenstrippt\the\dimexpr(#1)*2540/7227\relax}%
%
% pc 12/1
%
\def\texdimenpc#1{\expandafter\texdimenstrippt\the\dimexpr(#1)/12\relax}%
%
% nc 1370/107
%
\def\texdimennc#1{\expandafter\texdimenstrippt\the\dimexpr(#1)*107/1370\relax}%
%
% cc 14856/1157
%
\def\texdimencc#1{\expandafter\texdimenstrippt\the\dimexpr(#1)*1157/14856\relax}%
%
% cm 7227/254
%
\def\texdimencm#1{\expandafter\texdimenstrippt\the\dimexpr(#1)*254/7227\relax}%
%
% in 7227/100
%
\def\texdimenin#1{\expandafter\texdimenstrippt\the\dimexpr(#1)*100/7227\relax}%
%
% "no test" ceil()-based approach to up and down macros (release 1.0)
% =====================================================
%
% The notation <u/v> means u/v in numexpr, which does rounding
% away from zero. It is essential that the argument be >-0.5 else <x+1>
% not same as <x>+1. All formulae are overflow free.
%
% The comments are for T > 0.
%
% Roughly such an approach works for phi = a/b > 1, such that:
%
%     a*(2b+1)<2**31 if a is odd, <2**32 if a is even
%
% This is true for all core units with quite some margin, the one with
% largest a*b being phi=7227/2540 for "mm".
%
% Note: for a unit such as "ex" or "em" where morally b=65536=2**16,
% this limits to a<=16383 if a is odd and to a<=32766 if a is even.
% Thus the general \texdimenwithunit{dim1}{dim2} (which for dim2<1pt
% computes basically an "up" value) can *not imitate fully* this scheme.
%
% The macros and formulas in the comments were obtained from a template
% (see file generateupdownmacros.py at the project repository),
% and we could actually combine them into a generic macro handling
% general a/b (assuming above bounds are verified).
% But for the the sake of efficiency, this is "rolled-out" here unit per unit.
%
\def\texdimenuudownup_zero#1;{\z@\relax}%
\def\texdimenuudownup_neg#1-{-#1}%
% bp 803/800
% T = 803 k + r
% Zd = 800 k + 800 - <(1284003 - 1600 r)/1606>
\def\texdimenbpdown#1{\expandafter\texdimenstrippt\the\dimexpr
    \expandafter\texdimenbpdown_a\the\numexpr\dimexpr#1;%
}%
\def\texdimenbpdown_a#1{\texdimenzerominusfork
                        #1-\texdimenuudownup_zero
                        0#1\texdimenuudownup_neg
                         0-{}%
                        \krof \texdimenbpdown_b#1}%
\def\texdimenbpdown_b#1;{\expandafter\texdimenbpdown_c\the\numexpr(2*#1-803)/1606;#1;}%
\def\texdimenbpdown_c#1;#2;{\expandafter\texdimenbpdown_d\the\numexpr#2-803*#1;#1;}%
\def\texdimenbpdown_d#1;#2;{\numexpr800*#2+800-(1284003-1600*#1)/1606sp\relax}%
% Zu = 800 k + 800 + 1 - <(1285603 - 1600 r)/1606>
\def\texdimenbpup#1{\expandafter\texdimenstrippt\the\dimexpr
    \expandafter\texdimenbpup_a\the\numexpr\dimexpr#1;%
}%
\def\texdimenbpup_a#1{\texdimenzerominusfork
                        #1-\texdimenuudownup_zero
                        0#1\texdimenuudownup_neg
                         0-{}%
                      \krof \texdimenbpup_b#1}%
\def\texdimenbpup_b#1;{\expandafter\texdimenbpup_c\the\numexpr(2*#1-803)/1606;#1;}%
\def\texdimenbpup_c#1;#2;{\expandafter\texdimenbpup_d\the\numexpr#2-803*#1;#1;}%
\def\texdimenbpup_d#1;#2;{\numexpr800*#2+801-(1285603-1600*#1)/1606sp\relax}%
% nd 685/642
% T = 685 k + r
% Zd = 642 k + 642 - <(878941 - 1284 r)/1370>
\def\texdimennddown#1{\expandafter\texdimenstrippt\the\dimexpr
    \expandafter\texdimennddown_a\the\numexpr\dimexpr#1;%
}%
\def\texdimennddown_a#1{\texdimenzerominusfork
                        #1-\texdimenuudownup_zero
                        0#1\texdimenuudownup_neg
                         0-{}%
                        \krof \texdimennddown_b#1}%
\def\texdimennddown_b#1;{\expandafter\texdimennddown_c\the\numexpr(2*#1-685)/1370;#1;}%
\def\texdimennddown_c#1;#2;{\expandafter\texdimennddown_d\the\numexpr#2-685*#1;#1;}%
\def\texdimennddown_d#1;#2;{\numexpr642*#2+642-(878941-1284*#1)/1370sp\relax}%
% Zu = 642 k + 642 + 1 - <(880225 - 1284 r)/1370>
\def\texdimenndup#1{\expandafter\texdimenstrippt\the\dimexpr
    \expandafter\texdimenndup_a\the\numexpr\dimexpr#1;%
}%
\def\texdimenndup_a#1{\texdimenzerominusfork
                        #1-\texdimenuudownup_zero
                        0#1\texdimenuudownup_neg
                         0-{}%
                      \krof \texdimenndup_b#1}%
\def\texdimenndup_b#1;{\expandafter\texdimenndup_c\the\numexpr(2*#1-685)/1370;#1;}%
\def\texdimenndup_c#1;#2;{\expandafter\texdimenndup_d\the\numexpr#2-685*#1;#1;}%
\def\texdimenndup_d#1;#2;{\numexpr642*#2+643-(880225-1284*#1)/1370sp\relax}%
% dd 1238/1157
% T = 1238 k + r
% Zd = 1157 k + 1157 - <(1431828 - 1157 r)/1238>
\def\texdimendddown#1{\expandafter\texdimenstrippt\the\dimexpr
    \expandafter\texdimendddown_a\the\numexpr\dimexpr#1;%
}%
\def\texdimendddown_a#1{\texdimenzerominusfork
                        #1-\texdimenuudownup_zero
                        0#1\texdimenuudownup_neg
                         0-{}%
                        \krof \texdimendddown_b#1}%
\def\texdimendddown_b#1;{\expandafter\texdimendddown_c\the\numexpr(#1-619)/1238;#1;}%
\def\texdimendddown_c#1;#2;{\expandafter\texdimendddown_d\the\numexpr#2-1238*#1;#1;}%
\def\texdimendddown_d#1;#2;{\numexpr1157*#2+1157-(1431828-1157*#1)/1238sp\relax}%
% Zu = 1157 k + 1157 + 1 - <(1432985 - 1157 r)/1238>
\def\texdimenddup#1{\expandafter\texdimenstrippt\the\dimexpr
    \expandafter\texdimenddup_a\the\numexpr\dimexpr#1;%
}%
\def\texdimenddup_a#1{\texdimenzerominusfork
                        #1-\texdimenuudownup_zero
                        0#1\texdimenuudownup_neg
                         0-{}%
                      \krof \texdimenddup_b#1}%
\def\texdimenddup_b#1;{\expandafter\texdimenddup_c\the\numexpr(#1-619)/1238;#1;}%
\def\texdimenddup_c#1;#2;{\expandafter\texdimenddup_d\the\numexpr#2-1238*#1;#1;}%
\def\texdimenddup_d#1;#2;{\numexpr1157*#2+1158-(1432985-1157*#1)/1238sp\relax}%
% mm 7227/2540
% T = 7227 k + r
% Zd = 2540 k + 2540 - <(36715307 - 5080 r)/14454>
\def\texdimenmmdown#1{\expandafter\texdimenstrippt\the\dimexpr
    \expandafter\texdimenmmdown_a\the\numexpr\dimexpr#1;%
}%
\def\texdimenmmdown_a#1{\texdimenzerominusfork
                        #1-\texdimenuudownup_zero
                        0#1\texdimenuudownup_neg
                         0-{}%
                        \krof \texdimenmmdown_b#1}%
\def\texdimenmmdown_b#1;{\expandafter\texdimenmmdown_c\the\numexpr(2*#1-7227)/14454;#1;}%
\def\texdimenmmdown_c#1;#2;{\expandafter\texdimenmmdown_d\the\numexpr#2-7227*#1;#1;}%
\def\texdimenmmdown_d#1;#2;{\numexpr2540*#2+2540-(36715307-5080*#1)/14454sp\relax}%
% Zu = 2540 k + 2540 + 1 - <(36720387 - 5080 r)/14454>
\def\texdimenmmup#1{\expandafter\texdimenstrippt\the\dimexpr
    \expandafter\texdimenmmup_a\the\numexpr\dimexpr#1;%
}%
\def\texdimenmmup_a#1{\texdimenzerominusfork
                        #1-\texdimenuudownup_zero
                        0#1\texdimenuudownup_neg
                         0-{}%
                      \krof \texdimenmmup_b#1}%
\def\texdimenmmup_b#1;{\expandafter\texdimenmmup_c\the\numexpr(2*#1-7227)/14454;#1;}%
\def\texdimenmmup_c#1;#2;{\expandafter\texdimenmmup_d\the\numexpr#2-7227*#1;#1;}%
\def\texdimenmmup_d#1;#2;{\numexpr2540*#2+2541-(36720387-5080*#1)/14454sp\relax}%
% pc 12/1
% T = 12 k + r
% Zd = 1 k + 1 - <(17 - 1 r)/12>
\def\texdimenpcdown#1{\expandafter\texdimenstrippt\the\dimexpr
    \expandafter\texdimenpcdown_a\the\numexpr\dimexpr#1;%
}%
\def\texdimenpcdown_a#1{\texdimenzerominusfork
                        #1-\texdimenuudownup_zero
                        0#1\texdimenuudownup_neg
                         0-{}%
                        \krof \texdimenpcdown_b#1}%
\def\texdimenpcdown_b#1;{\expandafter\texdimenpcdown_c\the\numexpr(#1-6)/12;#1;}%
\def\texdimenpcdown_c#1;#2;{\expandafter\texdimenpcdown_d\the\numexpr#2-12*#1;#1;}%
\def\texdimenpcdown_d#1;#2;{\numexpr#2+1-(17-#1)/12sp\relax}%
% Zu = 1 k + 1 + 1 - <(18 - 1 r)/12>
\def\texdimenpcup#1{\expandafter\texdimenstrippt\the\dimexpr
    \expandafter\texdimenpcup_a\the\numexpr\dimexpr#1;%
}%
\def\texdimenpcup_a#1{\texdimenzerominusfork
                        #1-\texdimenuudownup_zero
                        0#1\texdimenuudownup_neg
                         0-{}%
                      \krof \texdimenpcup_b#1}%
\def\texdimenpcup_b#1;{\expandafter\texdimenpcup_c\the\numexpr(#1-6)/12;#1;}%
\def\texdimenpcup_c#1;#2;{\expandafter\texdimenpcup_d\the\numexpr#2-12*#1;#1;}%
\def\texdimenpcup_d#1;#2;{\numexpr#2+2-(18-#1)/12sp\relax}%
% nc 1370/107
% T = 1370 k + r
% Zd = 107 k + 107 - <(147168 - 107 r)/1370>
\def\texdimenncdown#1{\expandafter\texdimenstrippt\the\dimexpr
    \expandafter\texdimenncdown_a\the\numexpr\dimexpr#1;%
}%
\def\texdimenncdown_a#1{\texdimenzerominusfork
                        #1-\texdimenuudownup_zero
                        0#1\texdimenuudownup_neg
                         0-{}%
                        \krof \texdimenncdown_b#1}%
\def\texdimenncdown_b#1;{\expandafter\texdimenncdown_c\the\numexpr(#1-685)/1370;#1;}%
\def\texdimenncdown_c#1;#2;{\expandafter\texdimenncdown_d\the\numexpr#2-1370*#1;#1;}%
\def\texdimenncdown_d#1;#2;{\numexpr107*#2+107-(147168-107*#1)/1370sp\relax}%
% Zu = 107 k + 107 + 1 - <(147275 - 107 r)/1370>
\def\texdimenncup#1{\expandafter\texdimenstrippt\the\dimexpr
    \expandafter\texdimenncup_a\the\numexpr\dimexpr#1;%
}%
\def\texdimenncup_a#1{\texdimenzerominusfork
                        #1-\texdimenuudownup_zero
                        0#1\texdimenuudownup_neg
                         0-{}%
                      \krof \texdimenncup_b#1}%
\def\texdimenncup_b#1;{\expandafter\texdimenncup_c\the\numexpr(#1-685)/1370;#1;}%
\def\texdimenncup_c#1;#2;{\expandafter\texdimenncup_d\the\numexpr#2-1370*#1;#1;}%
\def\texdimenncup_d#1;#2;{\numexpr107*#2+108-(147275-107*#1)/1370sp\relax}%
% cc 14856/1157
% T = 14856 k + r
% Zd = 1157 k + 1157 - <(17194663 - 1157 r)/14856>
\def\texdimenccdown#1{\expandafter\texdimenstrippt\the\dimexpr
    \expandafter\texdimenccdown_a\the\numexpr\dimexpr#1;%
}%
\def\texdimenccdown_a#1{\texdimenzerominusfork
                        #1-\texdimenuudownup_zero
                        0#1\texdimenuudownup_neg
                         0-{}%
                        \krof \texdimenccdown_b#1}%
\def\texdimenccdown_b#1;{\expandafter\texdimenccdown_c\the\numexpr(#1-7428)/14856;#1;}%
\def\texdimenccdown_c#1;#2;{\expandafter\texdimenccdown_d\the\numexpr#2-14856*#1;#1;}%
\def\texdimenccdown_d#1;#2;{\numexpr1157*#2+1157-(17194663-1157*#1)/14856sp\relax}%
% Zu = 1157 k + 1157 + 1 - <(17195820 - 1157 r)/14856>
\def\texdimenccup#1{\expandafter\texdimenstrippt\the\dimexpr
    \expandafter\texdimenccup_a\the\numexpr\dimexpr#1;%
}%
\def\texdimenccup_a#1{\texdimenzerominusfork
                        #1-\texdimenuudownup_zero
                        0#1\texdimenuudownup_neg
                         0-{}%
                      \krof \texdimenccup_b#1}%
\def\texdimenccup_b#1;{\expandafter\texdimenccup_c\the\numexpr(#1-7428)/14856;#1;}%
\def\texdimenccup_c#1;#2;{\expandafter\texdimenccup_d\the\numexpr#2-14856*#1;#1;}%
\def\texdimenccup_d#1;#2;{\numexpr1157*#2+1158-(17195820-1157*#1)/14856sp\relax}%
% cm 7227/254
% T = 7227 k + r
% Zd = 254 k + 254 - <(3678035 - 508 r)/14454>
\def\texdimencmdown#1{\expandafter\texdimenstrippt\the\dimexpr
    \expandafter\texdimencmdown_a\the\numexpr\dimexpr#1;%
}%
\def\texdimencmdown_a#1{\texdimenzerominusfork
                        #1-\texdimenuudownup_zero
                        0#1\texdimenuudownup_neg
                         0-{}%
                        \krof \texdimencmdown_b#1}%
\def\texdimencmdown_b#1;{\expandafter\texdimencmdown_c\the\numexpr(2*#1-7227)/14454;#1;}%
\def\texdimencmdown_c#1;#2;{\expandafter\texdimencmdown_d\the\numexpr#2-7227*#1;#1;}%
\def\texdimencmdown_d#1;#2;{\numexpr254*#2+254-(3678035-508*#1)/14454sp\relax}%
% Zu = 254 k + 254 + 1 - <(3678543 - 508 r)/14454>
\def\texdimencmup#1{\expandafter\texdimenstrippt\the\dimexpr
    \expandafter\texdimencmup_a\the\numexpr\dimexpr#1;%
}%
\def\texdimencmup_a#1{\texdimenzerominusfork
                        #1-\texdimenuudownup_zero
                        0#1\texdimenuudownup_neg
                         0-{}%
                      \krof \texdimencmup_b#1}%
\def\texdimencmup_b#1;{\expandafter\texdimencmup_c\the\numexpr(2*#1-7227)/14454;#1;}%
\def\texdimencmup_c#1;#2;{\expandafter\texdimencmup_d\the\numexpr#2-7227*#1;#1;}%
\def\texdimencmup_d#1;#2;{\numexpr254*#2+255-(3678543-508*#1)/14454sp\relax}%
% in 7227/100
% T = 7227 k + r
% Zd = 100 k + 100 - <(1452427 - 200 r)/14454>
\def\texdimenindown#1{\expandafter\texdimenstrippt\the\dimexpr
    \expandafter\texdimenindown_a\the\numexpr\dimexpr#1;%
}%
\def\texdimenindown_a#1{\texdimenzerominusfork
                        #1-\texdimenuudownup_zero
                        0#1\texdimenuudownup_neg
                         0-{}%
                        \krof \texdimenindown_b#1}%
\def\texdimenindown_b#1;{\expandafter\texdimenindown_c\the\numexpr(2*#1-7227)/14454;#1;}%
\def\texdimenindown_c#1;#2;{\expandafter\texdimenindown_d\the\numexpr#2-7227*#1;#1;}%
\def\texdimenindown_d#1;#2;{\numexpr#200+100-(1452427-2*#100)/14454sp\relax}%
% Zu = 100 k + 100 + 1 - <(1452627 - 200 r)/14454>
\def\texdimeninup#1{\expandafter\texdimenstrippt\the\dimexpr
    \expandafter\texdimeninup_a\the\numexpr\dimexpr#1;%
}%
\def\texdimeninup_a#1{\texdimenzerominusfork
                        #1-\texdimenuudownup_zero
                        0#1\texdimenuudownup_neg
                         0-{}%
                      \krof \texdimeninup_b#1}%
\def\texdimeninup_b#1;{\expandafter\texdimeninup_c\the\numexpr(2*#1-7227)/14454;#1;}%
\def\texdimeninup_c#1;#2;{\expandafter\texdimeninup_d\the\numexpr#2-7227*#1;#1;}%
\def\texdimeninup_d#1;#2;{\numexpr#200+101-(1452627-2*#100)/14454sp\relax}%
%
% Mathematics ("both in and cm", added at release 0.9)
% ===========
%
% Let a and b be two non-negative integers such that U = floor(a 7227/100) = 
% floor(b 7227/254).  It can be proven that a=50k, b=127k for some integer k.
% The proof is left to reader.  So U = floor(7227 k /2) for some k.
%
% Let's now find the largest such U <= T. So U = floor(k 7227/2)<= T which is
% equivalent (as k is integer) to k 7227/2 <= T + 1/2, i.e.
%
%     kmax = floor((2T+1)/7227)
%
% If we used for x>0 the formula floor(x)=round(x-1/2)=<x-1/2> we would end
% up basically with some 4T hence overflow problems even in \numexpr.
% Here I used <.> to denote rounding in the sense of \numexpr. It is not
% 1-periodical due to how negative inputs are handled, but here x-1/2>-1/2.
%
% The following lemma holds: let T be a non-negative integer then
%
%     floor((2T+1)/7227) = <(2T - 3612)/7227>
%
% So we can compute this k, hence get a=50k, b=127k, all within \numexpr and
% avoiding overflow.
%
% Implementation
% ==============
%
% Regarding the output in pt or sp, we seem to need floor(k 7227/2).
% The computation of floor(k 7227/2) as <(7227 k - 1)/2> would require to
% check if k==0 so we do it rather as <(7227 k + 1)/2> - 1.  No overflow
% can arise as k = 297147 for \maxdimen, and then 7227 k = 2**31 - 2279 and
% there is ample room for 7227k+1 using \numexpr.
%
% But this step, as well as initial step to get kmax will require to separate
% handling of negative input from positive one.
%
% Alternative
% -----------
%
% For non-negative T we can compute U = ((T+1)/7227)*7227. If U <= T keep it,
% else if U > T, replace it by U - 3614. This is alternative road to the maximal
% floor(k 7227/2) at most equal to T.
%
% There is some slight under-efficiency to share macros across the 3 end targets
% as I added one layer of parentheses.
\def\texdimenbothincm#1{\expandafter\texdimenstrippt\the\dimexpr
                        \expandafter\texdimenboth_a\the\numexpr\dimexpr#1;127);}%
\def\texdimenbothcmin#1{\expandafter\texdimenstrippt\the\dimexpr
                        \expandafter\texdimenboth_a\the\numexpr\dimexpr#1;50);}%
\def\texdimenbothincmpt#1{\expandafter\texdimenstrippt\the\dimexpr
                          \expandafter\texdimenboth_a\the\numexpr\dimexpr#1;7227+1)/2-1;}%
\let\texdimenbothcminpt\texdimenbothincmpt
\def\texdimenboth_a#1{\if-#1\texdimenboth_neg\fi\texdimenboth_b#1}%
% The opening parenthesis ( is closed in #2, it was added to share "pt" output
% with the two others
\def\texdimenboth_b#1;#2;{\numexpr(((2*#1-3612)/7227)*#2sp\relax}%
% negative branch. This is expanded in a \dimexpr so we can insert the -
% in front of the \numexpr.
% #1 is here \texdimenboth_b
\def\texdimenboth_neg\fi#1-#2;#3;{\fi-\numexpr(((2*#2-3612)/7227)*#3sp\relax}%
%
% \texdimenbothincmsp is done separately as I found no easy way to share
% its macros with the others; alternative would have been to make it the
% core, and derive the others from it, (\texdimencm{\texdimenbothincmsp{...}sp})
% but then they would be less efficient than their current versions.
% (it is a bit ironical to worry about not creating too many macros
% in such a small package, by the way)
\def\texdimenbothincmsp#1{\the\numexpr\expandafter\texdimenbothsp_a\the\numexpr\dimexpr#1;}%
\def\texdimenbothsp_a#1{\if-#1\texdimenbothsp_neg\fi\texdimenbothsp_b#1}%
\def\texdimenbothsp_b#1;{(((2*#1-3612)/7227)*7227+1)/2-1\relax}%
% we need to regrab here or to add a \numexpr..\relax layer to
% \texdimenbothsp_b (parentheses could do but using 0-(...) syntax)
% finally doing the job of \texdimenbothsp_b directly
\def\texdimenbothsp_neg\fi#1-#2;{\fi-\numexpr(((2*#2-3612)/7227)*7227+1)/2-1\relax\relax}%
%
\let\texdimenbothcminsp\texdimenbothincmsp
%
% Mathematics and Algorithm ("both mm and bp", added at release 1.0)
% =========================
% We start from a dimension expressed in sp unit, "T sp". Assume T positive.
% We know how to get largest "X sp <= T sp" which is exactly expressible
% in mm unit
% i.e. can be written X=trunc(a 7227/2540) for some non-negative integer a.
% We want to achieve X=trunc(b 803/800) for some b.
%
%    Only the congruence of X modulo 803 matters for this.
%    It turns out that the mod 803 impossible values are 267, 535, 802.
%    As pointed out by Ruixi Zhang on the package repo issue #10,
%    when a<--a+2540, X increases by 7227=9*803 hence the value
%    modulo 803 does not change. Thus only "a modulo 2540" matters
%    to check if X(a) is attainable with bp unit. Ruixi Zhang found by
%    brute force that there are modulo 2540 nine excluded a-values
%
% Rather than checking if "a mod. 2540" avoids the 9 Ruixi Zhang values
% or if "X mod. 803" avoids  267, 535, 802, we will simply basically
% check if X sp = \texdimenbp{X sp}bp, as this approach is probably
% about the same cost or even less than computing "X mod. 803" and
% correspondingly branching.
%
% The key is that if "a" is bad, then "a-1" is automatically good as
% pointed out by R.Z. on #10, which can be seen without knowing the 9
% bad congruences, simply by noticing that a<--a-1 modifies X either to
% X-2 or X-3, so if X was bad certainly the new one is not.
%
% Once "a" has gotten its final value, we apply "\the\dimexpr a sp
% = D pt" trick to recover the D such that "D mm" gives rise to the found
% dimension.  We go via this "Dmm" intermediary also to express the final
% result as "X sp", because anyhow the "X" we worked with and had in
% our token stream has to be recomputed if a<--a-1, so lets always
% recompute it from final "a", and this goes via "D mm" (but see
% the paragraph MEMO for alternative for this trunc(a 7227/2540) step).
%
% I will copy here the style I used for bothincm expansion triggering
% via an already positioned \dimexpr waiting to output final result.
\def\texdimenbothbpmm#1{\expandafter\texdimenstrippt\the\dimexpr
                        \expandafter\texdimenbothbpmm_fork\the\numexpr\dimexpr#1;}%
\def\texdimenbothbpmm_fork#1{\texdimenzerominusfork
                             #1-\texdimenbothbpmm_zero
                             0#1\texdimenbothbpmm_neg
                             0-\texdimenbothbpmm_a
                             \krof#1}%
% because this is *inside* a pre-positioned \dimexpr, we don't have
% to worry about zero output ending up as -0.0
\def\texdimenbothbpmm_neg-{-\texdimenbothbpmm_a}%
\def\texdimenbothbpmm_zero#1;{\z@\relax}%
% now, find X sp <= T sp maximal and expressible in mm unit
% it will be X=trunc(a 7227/2540), we first get a candidate for "a"
\def\texdimenbothbpmm_a#1;%
    {\expandafter\texdimenbothbpmm_b\the\numexpr#1*2540/7227;#1;}%
% we get in a single line the X from this candidate, hijacking TeX's
% built-in *7227/2540... the "MEMO" above explains one could do this
% purely within \numexpr, working around its division rounds, and
% avoiding overflow, but I suspect this would be more costly.
\def\texdimenbothbpmm_b#1;{\expandafter\texdimenbothbpmm_c
    \the\numexpr\dimexpr\expandafter\texdimenstrippt\the\dimexpr#1spmm;#1;}%
% now we have X;a;T;
\def\texdimenbothbpmm_c#1;#2;#3;{%
% If X>T, our candidate "a=#2" must be decreased by 1 and we go to _ca
% The original #3 is not needed anymore
    \ifnum#1>#3 \expandafter\texdimenbothbpmm_ca\fi
% Else we decide whether it is "a" or "a-1" we must use. I preferred
% to induce a re-grabbing cost here, rather than have \texdimenbothbpmm_ca
% re-grab its arguments from \texdimenbothbpmm_d replacement text.
    \texdimenbothbpmm_d#1;#2;%
}%
% Here, dynamically at the time of the concluding \dimexpr, we
% check if X sp is expressible in bp unit and then use "a" or "a-1"
% accordingly
\def\texdimenbothbpmm_d#1;#2;{#2sp%
    \ifnum\dimexpr
    \expandafter\texdimenstrippt\the\dimexpr\numexpr(2*#1+1)*400/803spbp=#1
    \else-1sp\fi
% and a \relax to stop the concluding \dimexpr
    \relax
}%
% Here we must decrease "a=#2" by 1, recompute X=#1, then loop
% back to \texdimenbothbpmm_d. Hesitation between forcing a
% re-grab or doing it in one step with the subtraction of 1 done twice
\def\texdimenbothbpmm_ca\texdimenbothbpmm_d#1;#2;%
   {\expandafter\texdimenbothbpmm_cb\the\numexpr#2-1;}%
\def\texdimenbothbpmm_cb#1;{%
    \expandafter\texdimenbothbpmm_d
    \the\numexpr\dimexpr\expandafter\texdimenstrippt\the\dimexpr#1spmm;#1;%
}%
% done...
% now the lazy way for \texdimenbothmmbp
\def\texdimenbothmmbp#1{\expandafter\texdimenstrippt\the\dimexpr
    \expandafter\texdimenbothmmbp_a\the\numexpr\dimexpr\texdimenbothbpmm{#1}mm;}%
% or remove the + and do \if-#1-\else+\fi1 ?
% If zero at this stage, we will correctly get 0.0 in the end
\def\texdimenbothmmbp_a#1#2;{\numexpr(2*#1#2+\if-#1-\fi1)*400/803sp\relax}%
% \texdimenbothbpmmpt and its alias \texdimenbothmmbppt
\def\texdimenbothbpmmpt#1{\texdimenpt{\texdimenbothbpmm{#1}mm}}%
\let\texdimenbothmmbppt\texdimenbothbpmmpt
% \texdimenbothbpmmsp and its alias \texdimenbothmmbpsp
\def\texdimenbothbpmmsp#1{\the\numexpr\dimexpr\texdimenbothbpmm{#1}mm\relax\relax}%
\let\texdimenbothmmbpsp\texdimenbothbpmmsp
%
% Mathematics (\texdimenwithunit, added at release 0.99, modified at 1.0)
% ===========
%
% As explained in the README.md, the ex and em units are
% handled by TeX as if multiplying by a conversion factor f/65536
% (here f sp = 1ex resp. = 1em).
% In particular, for any decimal D, input "D em" is handled the exact
% same way as input "D\dimexpr 1em\relax"; this is not
% the case for the core units except for pt and pc (and sp), whose
% conversion factors are the sole ones with a power of 2 denominator
% (respectively 1, 1, and 65536).  The further difference is that
% for the core units apart from sp, the conversion factor is >1.
%
% We assume for this discussion T is non-negative.
% If f/65536 > 1, the analysis is as above : some dimensions T sp
% are not attainable as D uu, but the formula
%     N=round((2T+1)*32768/f)
% will give a suitable decimal D via \the\dimexpr N sp\relax.
% (if T=0, we get N=0 as 32768/f<0.5)
% This D will let TeX convert D uu into T sp, if the dimension
% is attainable else it will be a closest match
% either from above or below (not necessarily nearest overall).
%
% If f/65536=1, attention that above formula would give N=1 for
% T=0 (was bug #4).
%
% If f/65536<1, all dimensions Tsp are attainable as D uu. Indeed
% D uu is parsed by TeX via N=round(D*65536), then T=trunc(N*phi),
% with phi=f/65536. Starting from T we need to find an N such that
% T/phi <= N< (T+1)/phi. 
%
% This is equivalent to ceil(T/phi)<= N < ceil((T+1)/phi)
%
% Now obsolete remark: let v=(T+0.5)/phi. As its
% distance to the extremities is 0.5/phi>0.5, (phi>1) its rounding M
% to an integer verifies automatically T/phi < M < (T+1)/phi, so
% is a valid candidate. This was used at 0.99 release.
% (it is funny that N=round((2T+1)*32768/f) works for all f>0
%  *except* f=65536).
%
% The 1.0 release chooses to implement the ceil(T/phi) formula rather as
% it is closer to naive expectation "dim1/dim2" of a division.
%
% It is not obvious to compute this ceil(T/phi) without overflow.
%
% Implementation
% ==============
%
% \texdimenwithunit{dim1}{dim2}
%
% First done at 0.99, then refactored at 1.0:
% - to add support for dim2<0pt
% - to handle differently the dim2<1pt case and make the output
%   closer to mathematical dim1/dim2
%
% To handle dim2<0pt, we simply simultaneously do
% dim1<-- (-dim1) and dim2<-- (-dim2).
%
% dim2=0pt is not intercepted and will cause division by zero low-level
% error.  Code comments below were not adjusted and handle only
% dim2>0pt.
%
% We first get f from dim2 and branch according to whether f>65536,
% or f<=65536.
% We will also need to check the sign of T (dim1=T sp).
% f>65536: we compute round((2T+1)*32768/f)
% f=65536: merged with f<65536 branch
% f<65536: we compute ceil(T*65536/f)
%
%          rationale: round((2T+1)*32768/f) which was used at 0.99
%          would be ok [if f<65536 not f=65536 ! cf #3, #4] 
%          also for this branch
%
%          BUT it diverges noticeably from naive expectation
%          dim1/dim2 "=" T*65536/f the more so when f is small.
%          See issue #16 and also the discussion at #13.
%
%          As was explained in comments to issue #2 which asked for a
%          \texdimenwithunit the ceil(T*65536/f) is the smallest
%          allowable choice
%
%          To avoid arithmetic overflow issues we first do the euclidean
%          division T = k f + r, 0<= r < f
%
%          The final result in sp units will be k*65536 + ceil(r * 65536/f)
%          We don't do this k*65536 explicitly as it may overflow
%          but output the decimal k + E where E is the conversion
%          to a decimal 0.ddddd or 1.0 of "ceil(r * 65536/f) sp"
%          (which is at most 65536sp=1pt so E is at most 1.0).
%
%          To compute the Euclidean quotient k in \numexpr we use there
%          (2T-f)/(2f) i.e. round((2T-f)/2f) = trunc(T/f)
%          as we are careful to never have T=0 inthere...
%
%          Computing C = ceil(r * 65536/f) in \numexpr is the delicate
%          part, as r can be as large as f-1 hence 65535 and r*65536 would
%          overflow. We could compute R=round(r*65536/f) ("scaling operation")
%          then C=R+1 if R*f-65536*r<0, else C=R.
%
%          The problem is then: how to get the sign of R*f-65536*r without
%          overflow? I considered various ways.
%
%          But in the end, opted for simply this:
%          - get R=round(r*65536/f) as \the\numexpr r*65536/f ("scaling" no overflow)
%          - hence get E pt=\the\dimexpr R sp
%          - let TeX compute E<f sp>: if it turns out < r sp,
%            then C=R+1,
%            else C=R. Done.
%
\def\texdimenwithunit#1#2{\expandafter\texdimenwithunit_i
% no premultiplication of dim1 by 2 as was done for technical
% reasons when dim2<1pt branch used round((2T+1)*32768/f)
    \the\numexpr\dimexpr#2\expandafter;\the\numexpr\dimexpr#1;%
}%
\def\texdimenwithunit_i#1{%
     \texdimengobtominus#1\texdimenwithunit_switchsigns-%
     \texdimenwithunit_j#1%
}%
\def\texdimenwithunit_switchsigns-\texdimenwithunit_j-#1;#2%
{%
% due to \texdimenwithunit_Bneg we can not simply prefix dim1
% with -, as -0 is bad there. So let's check also if #2 is 0
    \texdimenzerominusfork
      #2-\texdimenwithunit_Bzero % abusive double usage
      0#2\texdimenwithunit_j     % abusive shortcut
       0-{\texdimenwithunit_ic#2}%
    \krof
    #1;%
}%
\def\texdimenwithunit_ic#1#2;{\texdimenwithunit_j#2;-#1}%
\def\texdimenwithunit_j#1;#2{%
        \ifnum#1>\p@\texdimenwithunit_A\fi
        \texdimenwithunit_B#2#1;%
}%
% unit>1pt, handle this as for bp.
% Attention it would be wrong for unit=1pt!
\def\texdimenwithunit_A\fi\texdimenwithunit_B#1#2;#3;{\fi
    \expandafter\texdimenstrippt
    \the\dimexpr\numexpr(2*#1#3+\if-#1-\fi1)*32768/#2sp\relax
    % - fine if dim1>0, <0, or =0
    % - with *\p@ better but an early doubled dim2 would complicate 1pt
    % test and not sure if doing \p@/(2*#2) here advantageous
}%
% unit<=1pt.
% if dim1<0, simply negate result for dim1>0 as it can not possibly be 0.0
% Indeed T*65536/f will be at least 1 so its ceil also (in fact ceil
% will even be at least 2 if f<65536).
% The dim1=0 case must get filtered out due to way of calculating the
% "ceil" in \numexpr
\def\texdimenwithunit_B#1{\texdimenzerominusfork
                           #1-\texdimenwithunit_Bzero
                           0#1\texdimenwithunit_Bneg
                           0-\texdimenwithunit_Ba
                          \krof#1}%
\def\texdimenwithunit_Ba#1#2;#3;{%
    % no overflow possible from 2*#1#3=2*dim1
    \expandafter\texdimenwithunit_Bb\the\numexpr(2*#1#3-#2)/(2*#2);#1#3;#2;%
}%
% now k;T;f;. Get the remainder r=T-k*f
\def\texdimenwithunit_Bb#1;#2;#3;{%
    \expandafter\texdimenwithunit_Bc\the\numexpr#2-#1*#3;#1;#3;%
}%
% now r;k;f;. We can start \the\numexpr k+ ....
% and there we will need to get R=round(r*65536/f), Ept=Rsp,
% check if E"f sp"<"r sp", if yes replace R by R+1 else keep E etc.
\def\texdimenwithunit_Bc#1;#2;#3;{%
% \the\numexpr k+0.ddddd is handy because it can well be actually
% \the\numexpr k+1.0, now that we use ceil approach in this branch
    \the\numexpr#2+\expandafter\texdimenwithunit_Bd
                   \the\numexpr #1*\p@/#3;#1;#3;%
}%
% R;r;f;
\def\texdimenwithunit_Bd#1;{%
    \expandafter\texdimenwithunit_Be\the\dimexpr#1sp;#1;%
}%
% Ept;R;r;f;
% #1=0.ddd... or 1.0 but has no end marker hence the
% \texdimenfirstofone{#1} as in \texdimendown_d and \texdimenup_d
{\catcode`P 12\catcode`T 12
\lowercase{\gdef\texdimenwithunit_Be#1PT};#2;#3;#4;{%
    \ifdim#1\dimexpr#4sp<#3sp \texdimenwithunit_Bf{#2}\fi
    \texdimenfirstofone{#1}%
    }%
}%
% #2 is \fi. Add a dimen storage \onesp for 1sp?
\def\texdimenwithunit_Bf#1#2\texdimenfirstofone#3{#2%
    \expandafter\texdimenstrippt\the\dimexpr#1sp+1sp\relax
}%
% Here definitely not caring about f-expandability. Or efficiency.
\def\texdimenwithunit_Bneg-{-\texdimenwithunit_Ba{}}%
\def\texdimenwithunit_Bzero#1;#2;{0.0}%
\texdimensendinput
%! Local variables:
%! mode: TeX
%! fill-column: 1000
%! sentence-end-double-space: t
%! End:
