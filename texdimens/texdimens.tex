% This is file texdimens.tex, part of texdimens package, which
% is distributed under the LPPL 1.3c. Copyright (c) 2021 Jean-François B.
% 2021/11/?? v0.999
\edef\texdimensendinput{\endlinechar\the\endlinechar%
\catcode`\noexpand _=\the\catcode`\_%
\catcode`\noexpand @=\the\catcode`\@\relax\noexpand\endinput}%
\endlinechar13\relax%
\catcode`\_=11 \catcode`\@=11 % only for using \p@ (also \z@ now) of Plain. Check exists?
%
% Mathematics ("down" and "up" macros)
% ===========
%
% In the entire discussion here, "uu" stands for some core unit,
% or some unit corresponding to a dimension > 1pt. For the case
% of a unit corresponding to a dimension < 1pt, i.e. to
% \texdimenwithunit macro added at 0.99, refer to the
% comments of issue #2 on the tracker site.
%
% Is T sp attainable from unit "uu"?.
% If not, what is largest dimension < Tsp which is attainable?
% Here we suppose T>0.
%
% phi>1, psi=1/phi, psi<1.
%
%     U(N,phi)=trunc(N phi) is the strictly increasing sequence,
%     indexed by non-negative integers, of attainable dimensions.
%     (in sp unit)
%
%     U(N)<= T <  U(N+1)    iff    N = ceil((T+1)psi) - 1
%     U(M)<  T <= U(M+1)    iff    M = ceil(T psi)    - 1
%
% In other words:
%
% - the largest attainable dimension not exceeding T sp
%   is obtained via the integer "Zd = ceil((T+1)psi) - 1",
%   (i.e. find D with Zd=round(65536 D) then "D uu" is "down" approximation)
%
% - the smallest attainable dimension at least equal to T sp
%   is obtained from the integer "Zu = ceil(T psi)"
%
% The two "Z"'s are either equal (i.e. T is attained) or Zu=Zd+1.
%
% Spoiler 1: we will explain below that "R = round((T+0.5)psi)" has these
% properties
%
% - if T is attained then R=Zu=Zd
%
% - if T is not attained then R=Zu or R=Zd but we don't know which one
%   without further investigation.
%
% Spoiler 2: to compute Zu or Zd is difficult as ceil() is difficult in
% \numexpr. So basically for this task we compute R and check if it
% gives an approximant from above or below. It will appear that for
% psi>2 a simplification arises and round(T psi) can be used in place of
% R.
%
% Stumbling block
% ---------------
%
% The stumbling block is that computing "ceil((T+1)psi) - 1" without
% overflow is not obvious: yes \numexpr/\dimexpr allow so-called
% "scaling operations" but only in the "rounding up" variant.
%
% If we attempt computing the ceil(x) function via round(x+0.5)
%          (ATTENTION: if x>=0 is an integer this fails anyhow)
%
% For example with psi=100/7227 which corresponds to the unit "in",
% this necessitates evaluating:
%
%     P = round((((T+1)*200)+7227)/14454)
%
% But as far as I can tell currently, for this we need to be able
% to evaluate without overflow (T+1)*200+7227 and this limits to
% T's which are (roughly) such that 100 T is less than \maxdimen.
%
% Update (2021/11/07) 
% -------------------
%
% We could do the Euclidean division T = k*14454 + r
% then the P would be  re-expressed.
%
% Let's apply the idea rather on the original ceil((T+1)*100/7227).
% So we rather do Euclidean dvision T = k*7227 + r
%
% (T+1)*100/7227 becomes 100*k + (r+1)*100/7227 and we are reduced
% to obtain ceil(x) for x = n*100/7227, n=1+r.
%
% Here we have a nice situation 0 < x <= 100 as r+1 is at most 7227. Then
% 
% ceil(x) = 100 - floor(100 - x)
%         = 100 - (round(100 - x + 0.5) - 1)
%         = 101 - round(100 * (1 - n/7227) + 0.5)
%         = 101 - round((200 * (7227 - n) + 7227)/14454)
%
% To sum up. We can achieve the computation of Zd = ceil((T+1)*100/7227) - 1
% for T>0 without overflow in \numexpr this way:
%
%     k = floor(T/7227) = round(T/7227 - 0.5)
%                       = round((2*T - 7227) / 14454)
%
%     r = T - 7227 * k  = T modulo 7227
%
%     Zd = 100 * k + 100 - round( (201*7227 - 200*(r+1))/14454 )
%
% Everything here is computable within \numexpr with no potential overflow
% problem at all, and we don't even use "scaling" operations allowed by
% \numexpr (i.e. temporary extra precision for A*B/C).
%
% A work-around (i.e. the legacy approach)
% -------------
%
% The rest of the discussion is about an algorithm providing an
% alternative route to N={Zu or Zd}, using \numexpr/\dimexpr/TeX
% facilities. This was implemented in the first release of the package
% and I feel it involves less calculations than what would imply the
% 2021/11/7 update above (fully worked out for the "in" unit).
%
% However: for some units the computations of Zu and Zd will fail
% with "Dimension too large" error, when the input is very close to
% \maxdimen. Indeed the algorithm is based on computing (always
% without overflow) in an easy \numexpr way a candidate N which
% will be guaranteed to be either Zu or Zd. But to test which one
% it is, we have make a dimension evaluation which may raise
% overflow if N=Zu and trun(N*phi)>2**30-1=\maxdimen in sp unit.
%
% Let's return to the U(N)<= T < U(N+1) and U(M)< T <= U(M+1) equations.
%
% Recall in all of this T > 0. Either:
%
% case1:  M = N, i.e. T is not attainable, M=N < T psi < (T+1) psi <= N+1
% case2:  M = N - 1, i.e. T is attained, T psi <= N < (T+1) psi, T = trunc(N phi)
%
% Let X = round(T psi). And let Y = trunc(X phi). We will explain later
% how X and Y can be computed using \numexpr/\dimexpr/TeX.
%
% case1: X can be N or N+1. It will be N+1 iff Y > T.
% case2: X can be N or N-1. It will be N iff trunc((X+1)phi)>T.
%
% This is not convenient: if Y < T it could be that we are in case 2
% but to decide we must check if trunc((X+1) phi) = T or not, so
% this means a second computation.
%
% If psi < 0.5 (i.e. the unit is > 2pt)
% ------------
%
% The situation then simplifies:
%
% case1: (same)
% case2: X is necessarily N.
%
% Thus:
% a) compute X = round(T psi)
% b) compute Y = trunc(X phi) and test if Y > T. If true, we
%    were in case 1, replace X by X - 1, else we were either
%    in case 1 or case 2, and we leave X as it is.
% We have thus found N.
%
% The operation Y = trunc(X phi) can be achieved this way:
% i) use \the\dimexpr to convert X sp into D pt,
% ii) use \the\numexpr\dimexpr  to convert "D uu" into sp.
% These steps give Y.
%
% **** MEMO: this has advantage of brevity and hacks into
% **** TeX way of multiplying by such a factor say 7227/2540
% **** but with more \numexpr we could do this (with some pain
% **** as it rounds, not truncate): here prepare
% **** X as k 2540 + R, then Y = 7227*k + trunc(R*7227/2540)
% **** where the product R*7227<2540*7227 can be computed
% **** without overflow so we again have a Euclidean division
%
% This way we find the maximal dimension at most T sp exactly
% representable in "uu" unit.
%
% The computations of X and Y can be done independently of sign of T.
% But the final test has to be changed to Y < T if T < 0 and then
% one must replace X by X+1. So we must filter out the sign of the input.
%
% If the goal is only to find a decimal D such that "D uu" is
% exactly T sp in the case this is possible, then things are simpler
% because from X = round(T psi) we get D such as X sp is same as D pt
% and "D uu" will work.
% We don't have to take sign into account for this computation.
% But if T sp was not attainable we don't know if this X will give
% a D such that D uu < T sp or D uu > T sp.
%
% If 1 > psi > 0.5 (i.e. 1pt < unit < 2pt)
% ------------
%
% For example unit "bp" has phi=803/800.
%
% It is then not true that if T sp is attainable, the X = round(T psi)
% will always work.
%
% But it is true that R = round((T + 0.5) psi) will always work (T>0).
%
% Indeed we saw that "T sp" is attained iff there is an integer N
% such that T psi <= N < (T+1) psi. The distance of N to the mid-point
% (T + 0.5) psi is at most 0.5 psi. But psi < 1, so this distance is
% always < 0.5 and N must be R = round((T + 0.5) psi).
%
% This R=round((T+0.5) psi) can always be computed via \numexpr because 2T+1
% will not trigger arithmetic overflow.
%
% So this gives an approach to find a D such that "D uu" is exactly
% T sp when this is possible.
%
% If "T sp" is not attainable, we are in the N < T psi < (T+1) psi <= N+1
% situation then N < (T + 0.5) psi < N+1 and the rounding R can produce
% either N or N+1, i.e. either Zd or Zu in the notations of the introduction.
%
% We can decide what happened by computing Z = trunc(R phi).
% Z > T if and only if R was in fact N+1. So we can provide
% the largest attainable dimension smaller than Tsp as well
% as the smallest attainable one larger than Tsp.
%
% It is slightly less costly to compute X = round(T psi) than
% R = round((T + 0.5) psi),
% but if we then realize that trunc(X phi) < T  we do not yet know
% if trunc((X+1) phi) = T  or is > T. So we proceed via R, not X,
% to not have to make a second computation if a dimension comparison
% test goes awry.
%
% To recapitulate: we have our algorithm for all units to find out
% maximal dimension exactly attainable in "uu" unit and at most equal
% to (positive) T sp.
%
% Unfortunately the check that Y (in case psi < 0.5) or Z (in case psi >
% 0.5) verifies or not Y > T may trigger a Dimension too large error if
% T sp was near non-attainable \maxdimen. It turns out this sad
% situation happens only for the units `dd`, `nc`, and `in`, and T sp
% very close to \maxdimen (like for all units apart from `pt`, `bp`,
% `nd`, the \maxdimen is not attainable, and by bad luck for `dd`, `nc`,
% and `in`, the X will correspond to a decimal D such that Duu>\maxdimen
% is the nearest virtually attaible dimensions from above not from
% below; see the README.md for the tabulation of the maximal usable inputs).
%
% Regarding the \texdimen<uu> macros, and units with phi > 2, I
% hesitated using either the round((T+0.5)psi) or round(T psi), but for
% Tsp = \maxdimen, both formulas turned out to give the same result for
% all such units, so I chose for these \texdimen<uu> macros and the
% units with phi>2 to use the simpler round(T psi) which does not need
% to check the sign of T.
%
% For the "up" and "down" macros, we again use the round(T psi), but do
% have to check the sign anyhow. We could also have used the
% round((T+0.5)psi) which requires a sign check too, but it costs a bit
% more. It would have allowed though to share the same codebase for all
% units, here we have to prepare some slightly different shared macros
% for the first batch bp, nd, dd and the second batch mm, pc, nc, cc,
% cm, in.
%
% Implementation
% ==============
%
\def\texdimenfirstofone#1{#1}%
{\catcode`p 12\catcode`t 12
 \csname expandafter\endcsname\gdef\csname texdimenstrippt\endcsname#1pt{#1}}%
%
% down macros:
% for units with phi < 2:
\def\texdimendown_A#1{\if-#1\texdimendown_neg\fi\texdimendown_B#1}%
\def\texdimendown_B#1;#2;{\expandafter\texdimendown_c\the\numexpr(2*#1+1)#2;#1;}%
% for units with phi > 2:
\def\texdimendown_a#1{\if-#1\texdimendown_neg\fi\texdimendown_b#1}%
\def\texdimendown_b#1;#2;{\expandafter\texdimendown_c\the\numexpr#1#2;#1;}%
% shared macros:
\def\texdimendown_c#1;{\expandafter\texdimendown_d\the\dimexpr#1sp;#1;}%
{\catcode`P 12\catcode`T 12\lowercase{\gdef\texdimendown_d#1PT};#2;#3;#4;%
   {\ifdim#1#4>#3sp \texdimendown_e{#2}\fi\texdimenfirstofone{#1}}%
}%
% this #2 will be \fi
\def\texdimendown_e#1#2#3#4{#2\expandafter\texdimenstrippt\the\dimexpr\numexpr#1-1sp\relax}%
% negative branch:
% The problem here is that if input very small, output can be 0.0, and we
% do not want -0.0 as output.
% So let's do this somewhat brutally and non-efficiently.
% Anyhow, negative inputs are not our priority.
% #1 is \fi here and #2 is \texdimendown_b or _B:
\def\texdimendown_neg#1#2-#3;#4;#5;{#1\expandafter\texdimenstrippt\the\dimexpr-#2#3;#4;#5;pt\relax}%
%
% up macros:
\def\texdimenup_A#1{\if-#1\texdimenup_neg\fi\texdimenup_B#1}%
\def\texdimenup_B#1;#2;{\expandafter\texdimenup_c\the\numexpr(2*#1+1)#2;#1;}%
\def\texdimenup_a#1{\if-#1\texdimenup_neg\fi\texdimenup_b#1}%
\def\texdimenup_b#1;#2;{\expandafter\texdimenup_c\the\numexpr#1#2;#1;}%
\def\texdimenup_c#1;{\expandafter\texdimenup_d\the\dimexpr#1sp;#1;}%
{\catcode`P 12\catcode`T 12\lowercase{\gdef\texdimenup_d#1PT};#2;#3;#4;%
   {\ifdim#1#4<#3sp \texdimenup_e{#2}\fi\texdimenfirstofone{#1}}%
}%
% this #2 will be \fi
\def\texdimenup_e#1#2#3#4{#2\expandafter\texdimenstrippt\the\dimexpr\numexpr#1+1sp\relax}%
% negative branch:
% Here we can me more expeditive than for the "down" macros.
% But this breaks f-expandability.
% #1 will be \fi and #2 is \texdimenup_b or _B:
\def\texdimenup_neg#1#2-{#1-#2}%
%
% pt
%
\def\texdimenpt#1{\expandafter\texdimenstrippt\the\dimexpr#1\relax}%
%
% bp 7227/7200 = 803/800
%
\def\texdimenbp#1{\expandafter\texdimenbp_\the\numexpr\dimexpr#1;}%
\def\texdimenbp_#1#2;{%
    \expandafter\texdimenstrippt\the\dimexpr\numexpr(2*#1#2+\if-#1-\fi1)*400/803sp\relax
}%
% \texdimenbpdown: maximal dim exactly expressible in bp and at most equal to input
\def\texdimenbpdown#1{\expandafter\texdimendown_A\the\numexpr\dimexpr#1;*400/803;bp;}%
% \texdimenbpup: minimal dim exactly expressible in bp and at least equal to input
\def\texdimenbpup#1{\expandafter\texdimenup_A\the\numexpr\dimexpr#1;*400/803;bp;}%
%
% nd 685/642
%
\def\texdimennd#1{\expandafter\texdimennd_\the\numexpr\dimexpr#1;}%
\def\texdimennd_#1#2;{%
    \expandafter\texdimenstrippt\the\dimexpr\numexpr(2*#1#2+\if-#1-\fi1)*321/685sp\relax
}%
% \texdimennddown: maximal dim exactly expressible in nd and at most equal to input
\def\texdimennddown#1{\expandafter\texdimendown_A\the\numexpr\dimexpr#1;*321/685;nd;}%
% \texdimenndup: minimal dim exactly expressible in nd and at least equal to input
\def\texdimenndup#1{\expandafter\texdimenup_A\the\numexpr\dimexpr#1;*321/685;nd;}%
%
% dd 1238/1157
%
\def\texdimendd#1{\expandafter\texdimendd_\the\numexpr\dimexpr#1;}%
\def\texdimendd_#1#2;{%
    \expandafter\texdimenstrippt\the\dimexpr\numexpr(2*#1#2+\if-#1-\fi1)*1157/2476sp\relax
}%
% \texdimendddown: maximal dim exactly expressible in dd and at most equal to input
\def\texdimendddown#1{\expandafter\texdimendown_A\the\numexpr\dimexpr#1;*1157/2476;dd;}%
% \texdimenddup: minimal dim exactly expressible in dd and at least equal to input
\def\texdimenddup#1{\expandafter\texdimenup_A\the\numexpr\dimexpr#1;*1157/2476;dd;}%
%
% mm 7227/2540 phi now >2, use from here on the simpler approach
%
\def\texdimenmm#1{\expandafter\texdimenstrippt\the\dimexpr(#1)*2540/7227\relax}%
% \texdimenmmdown: maximal dim exactly expressible in mm and at most equal to input
\def\texdimenmmdown#1{\expandafter\texdimendown_a\the\numexpr\dimexpr#1;*2540/7227;mm;}%
% \texdimenmmup: minimal dim exactly expressible in mm and at least equal to input
\def\texdimenmmup#1{\expandafter\texdimenup_a\the\numexpr\dimexpr#1;*2540/7227;mm;}%
%
% pc 12/1
%
\def\texdimenpc#1{\expandafter\texdimenstrippt\the\dimexpr(#1)/12\relax}%
% \texdimenpcdown: maximal dim exactly expressible in pc and at most equal to input
\def\texdimenpcdown#1{\expandafter\texdimendown_a\the\numexpr\dimexpr#1;/12;pc;}%
% \texdimenpcup: minimal dim exactly expressible in pc and at least equal to input
\def\texdimenpcup#1{\expandafter\texdimenup_a\the\numexpr\dimexpr#1;/12;pc;}%
%
% nc 1370/107
%
\def\texdimennc#1{\expandafter\texdimenstrippt\the\dimexpr(#1)*107/1370\relax}%
% \texdimenncdown: maximal dim exactly expressible in nc and at most equal to input
\def\texdimenncdown#1{\expandafter\texdimendown_a\the\numexpr\dimexpr#1;*107/1370;nc;}%
% \texdimenncup: minimal dim exactly expressible in nc and at least equal to input
\def\texdimenncup#1{\expandafter\texdimenup_a\the\numexpr\dimexpr#1;*107/1370;nc;}%
%
% cc 14856/1157
%
\def\texdimencc#1{\expandafter\texdimenstrippt\the\dimexpr(#1)*1157/14856\relax}%
% \texdimenccdown: maximal dim exactly expressible in cc and at most equal to input
\def\texdimenccdown#1{\expandafter\texdimendown_a\the\numexpr\dimexpr#1;*1157/14856;cc;}%
% \texdimenccup: minimal dim exactly expressible in cc and at least equal to input
\def\texdimenccup#1{\expandafter\texdimenup_a\the\numexpr\dimexpr#1;*1157/14856;cc;}%
%
% cm 7227/254
%
\def\texdimencm#1{\expandafter\texdimenstrippt\the\dimexpr(#1)*254/7227\relax}%
% \texdimencmdown: maximal dim exactly expressible in cm and at most equal to input
\def\texdimencmdown#1{\expandafter\texdimendown_a\the\numexpr\dimexpr#1;*254/7227;cm;}%
% \texdimencmup: minimal dim exactly expressible in cm and at least equal to input
\def\texdimencmup#1{\expandafter\texdimenup_a\the\numexpr\dimexpr#1;*254/7227;cm;}%
%
% in 7227/100
%
\def\texdimenin#1{\expandafter\texdimenstrippt\the\dimexpr(#1)*100/7227\relax}%
% \texdimenindown: maximal dim exactly expressible in in and at most equal to input
\def\texdimenindown#1{\expandafter\texdimendown_a\the\numexpr\dimexpr#1;*100/7227;in;}%
% \texdimeninup: minimal dim exactly expressible in in and at least equal to input
\def\texdimeninup#1{\expandafter\texdimenup_a\the\numexpr\dimexpr#1;*100/7227;in;}%
% both in and cm
% Mathematics ("both" macros)
% ===========
%
% Let a and b be two non-negative integers such that U = floor(a 7227/100) = 
% floor(b 7227/254).  It can be proven that a=50k, b=127k for some integer k.
% The proof is left to reader.  So U = floor(7227 k /2) for some k.
%
% Let's now find the largest such U <= T. So U = floor(k 7227/2)<= T which is
% equivalent (as k is integer) to k 7227/2 <= T + 1/2, i.e.
%
%     kmax = floor((2T+1)/7227)
%
% If we used for x>0 the formula floor(x)=round(x-1/2)=<x-1/2> we would end
% up basically with some 4T hence overflow problems even in \numexpr.
% Here I used <.> to denote rounding in the sense of \numexpr. It is not
% 1-periodical due to how negative inputs are handled, but here x-1/2>-1/2.
%
% The following lemma holds: let T be a non-negative integer then
%
%     floor((2T+1)/7227) = <(2T - 3612)/7227>
%
% So we can compute this k, hence get a=50k, b=127k, all within \numexpr and
% avoiding overflow.
%
% Implementation
% ==============
%
% Regarding the output in pt or sp, we seem to need floor(k 7227/2).
% The computation of floor(k 7227/2) as <(7227 k - 1)/2> would require to
% check if k==0 so we do it rather as <(7227 k + 1)/2> - 1.  No overflow
% can arise as k = 297147 for \maxdimen, and then 7227 k = 2**31 - 2279 and
% there is ample room for 7227k+1 using \numexpr.
%
% But this step, as well as initial step to get kmax will require to separate
% handling of negative input from positive one.
%
% Alternative
% -----------
%
% For non-negative T we can compute U = ((T+1)/7227)*7227. If U <= T keep it,
% else if U > T, replace it by U - 3614. This is alternative road to the maximal
% floor(k 7227/2) at most equal to T.
%
% There is some slight under-efficiency to share macros across the 3 end targets
% as I added one layer of parentheses.
\def\texdimenbothincm#1{\expandafter\texdimenstrippt\the\dimexpr
                        \expandafter\texdimenboth_a\the\numexpr\dimexpr#1;127);}%
\def\texdimenbothcmin#1{\expandafter\texdimenstrippt\the\dimexpr
                        \expandafter\texdimenboth_a\the\numexpr\dimexpr#1;50);}%
\def\texdimenbothincmpt#1{\expandafter\texdimenstrippt\the\dimexpr
                          \expandafter\texdimenboth_a\the\numexpr\dimexpr#1;7227+1)/2-1;}%
\let\texdimenbothcminpt\texdimenbothincmpt
\def\texdimenboth_a#1{\if-#1\texdimenboth_neg\fi\texdimenboth_b#1}%
% The opening parenthesis ( is closed in #2, it was added to share "pt" output
% with the two others
\def\texdimenboth_b#1;#2;{\numexpr(((2*#1-3612)/7227)*#2sp\relax}%
% negative branch. This is expanded in a \dimexpr so we can insert the -
% in front of the \numexpr.
% #1 is \fi here and #2 is \texdimenboth_b
\def\texdimenboth_neg#1#2-#3;#4;{#1-\numexpr(((2*#3-3612)/7227)*#4sp\relax}%
%
% \texdimenbothincmsp is done separately as I found no easy way to share
% its macros with the others; alternative would have been to make it the
% core, and derive the others from it, (\texdimencm{\texdimenbothincmsp{...}sp})
% but then they would be less efficient than their current versions.
% (it is a bit ironical to worry about not creating too many macros
% in such a small package, by the way)
\def\texdimenbothincmsp#1{\the\numexpr\expandafter\texdimenbothsp_a\the\numexpr\dimexpr#1;}%
\def\texdimenbothsp_a#1{\if-#1\texdimenbothsp_neg\fi\texdimenbothsp_b#1}%
\def\texdimenbothsp_b#1;{(((2*#1-3612)/7227)*7227+1)/2-1\relax}%
% #1 is \fi
% we need to regrab here or to add a \numexpr..\relax layer to
% \texdimenbothsp_b (parentheses could do but using 0-(...) syntax)
% finally doing the job of \texdimenbothsp_b directly
\def\texdimenbothsp_neg#1#2-#3;{#1-\numexpr(((2*#3-3612)/7227)*7227+1)/2-1\relax\relax}%
%
\let\texdimenbothcminsp\texdimenbothincmsp
% both mm and bp, cf issue #10 on tracker
% Mathematics and Algorithm
% =========================
% We start from a dimension expressed in sp unit, "T sp". Assume T positive.
% We know how to get largest "X sp <= T sp" which is exactly expressible
% in mm unit
% i.e. can be written X=trunc(a 7227/2540) for some non-negative integer a.
% We want to achieve X=trunc(b 803/800) for some b.
%
%    Only the congruence of X modulo 803 matters for this.
%    It turns out that the mod 803 impossible values are 267, 535, 802.
%    As pointed out by Ruixi Zhang on the package repo issue #10,
%    when a<--a+2540, X increases by 7227=9*803 hence the value
%    modulo 803 does not change. Thus only "a modulo 2540" matters
%    to check if X(a) is attainable with bp unit. Ruixi Zhang found by
%    brute force that there are modulo 2540 nine excluded a-values
%
% Rather than checking if "a mod. 2540" avoids the 9 Ruixi Zhang values
% or if "X mod. 803" avoids  267, 535, 802, we will simply basically
% check if X sp = \texdimenbp{X sp}bp, as this approach is probably
% about the same cost or even less than computing "X mod. 803" and
% correspondingly branching.
%
% The key is that if "a" is bad, then "a-1" is automatically good as
% pointed out by R.Z. on #10, which can be seen without knowing the 9
% bad congruences, simply by noticing that a<--a-1 modifies X either to
% X-2 or X-3, so if X was bad certainly the new one is not.
%
% Once "a" has gotten its final value, we apply "\the\dimexpr a sp
% = D pt" trick to recover the D such that "D mm" gives rise to the found
% dimension.  We go via this "Dmm" intermediary also to express the final
% result as "X sp", because anyhow the "X" we worked with and had in
% our token stream has to be recomputed if a<--a-1, so lets always
% recompute it from final "a", and this goes via "D mm" (but see
% the paragraph MEMO for alternative for this trunc(a 7227/2540) step).
%
% I will copy here the style I used for bothincm expansion triggering
% via an already positioned \dimexpr waiting to output final result.
\def\texdimenbothbpmm#1{\expandafter\texdimenstrippt\the\dimexpr
                        \expandafter\texdimenbothbpmm_fork\the\numexpr\dimexpr#1;}%
% the \texdimenzerominusfork is defined in the \texdimenwithunit section
\def\texdimenbothbpmm_fork#1{\texdimenzerominusfork
                             #1-\texdimenbothbpmm_zero
                             0#1\texdimenbothbpmm_neg
                             0-\texdimenbothbpmm_a
                             \krof#1}%
% because this is *inside* a pre-positioned \dimexpr, we don't have
% to worry about zero output ending up as -0.0
\def\texdimenbothbpmm_neg-{-\texdimenbothbpmm_a}%
\def\texdimenbothbpmm_zero#1;{\z@\relax}%
% now, find X sp <= T sp maximal and expressible in mm unit
% it will be X=trunc(a 7227/2540), we first get a candidate for "a"
\def\texdimenbothbpmm_a#1;%
    {\expandafter\texdimenbothbpmm_b\the\numexpr#1*2540/7227;#1;}%
% we get in a single line the X from this candidate, hijacking TeX's
% built-in *7227/2540... the "MEMO" above explains one could do this
% purely within \numexpr, working around its division rounds, and
% avoiding overflow, but I suspect this would be more costly.
\def\texdimenbothbpmm_b#1;{\expandafter\texdimenbothbpmm_c
    \the\numexpr\dimexpr\expandafter\texdimenstrippt\the\dimexpr#1spmm;#1;}%
% now we have X;a;T;
\def\texdimenbothbpmm_c#1;#2;#3;{%
% If X>T, our candidate "a=#2" must be decreased by 1 and we go to _ca
% The original #3 is not needed anymore
    \ifnum#1>#3 \expandafter\texdimenbothbpmm_ca\fi
% Else we decide whether it is "a" or "a-1" we must use. I preferred
% to induce a re-grabbing cost here, rather than have \texdimenbothbpmm_ca
% re-grab its arguments from \texdimenbothbpmm_d replacement text.
    \texdimenbothbpmm_d#1;#2;%
}%
% Here, dynamically at the time of the concluding \dimexpr, we
% check if X sp is expressible in bp unit and then use "a" or "a-1"
% accordingly
\def\texdimenbothbpmm_d#1;#2;{#2sp%
    \ifnum\dimexpr
    \expandafter\texdimenstrippt\the\dimexpr\numexpr(2*#1+1)*400/803spbp=#1
    \else-1sp\fi
% and a \relax to stop the concluding \dimexpr
    \relax
}%
% Here we must decrease "a=#2" by 1, recompute X=#1, then loop
% back to \texdimenbothbpmm_d. Hesitation between forcing a
% re-grab or doing it in one step with the subtraction of 1 done twice
\def\texdimenbothbpmm_ca\texdimenbothbpmm_d#1;#2;%
   {\expandafter\texdimenbothbpmm_cb\the\numexpr#2-1;}%
\def\texdimenbothbpmm_cb#1;{%
    \expandafter\texdimenbothbpmm_d
    \the\numexpr\dimexpr\expandafter\texdimenstrippt\the\dimexpr#1spmm;#1;%
}%
% done...
% now the lazy way for \texdimenbothmmbp
\def\texdimenbothmmbp#1{\expandafter\texdimenstrippt\the\dimexpr
    \expandafter\texdimenbothmmbp_a\the\numexpr\dimexpr\texdimenbothbpmm{#1}mm;}%
% or remove the + and do \if-#1-\else+\fi1 ?
% If zero at this stage, we will correctly get 0.0 in the end
\def\texdimenbothmmbp_a#1#2;{\numexpr(2*#1#2+\if-#1-\fi1)*400/803sp\relax}%
% \texdimenbothbpmmpt and its alias \texdimenbothmmbppt
\def\texdimenbothbpmmpt#1{\texdimenpt{\texdimenbothbpmm{#1}mm}}%
\let\texdimenbothmmbppt\texdimenbothbpmmpt
% \texdimenbothbpmmsp and its alias \texdimenbothmmbpsp
\def\texdimenbothbpmmsp#1{\the\numexpr\dimexpr\texdimenbothbpmm{#1}mm\relax\relax}%
\let\texdimenbothmmbpsp\texdimenbothbpmmsp
% (\texdimenwithunit, added at 0.99)
% Mathematics
% ===========
%
% As explained in the README.md, the ex and em units are
% handled by TeX as if multiplying by a conversion factor f/65536
% (here f sp = 1ex resp. = 1em).
% In particular, for any decimal D, input "D em" is handled the exact
% same way as input "D\dimexpr 1em\relax"; this is not
% the case for the core units except for pt and pc (and sp), whose
% conversion factors are the sole ones with a power of 2 denominator
% (respectively 1, 1, and 65536).  The further difference is that
% for the core units apart from sp, the conversion factor is >1.
%
% We assume for this discussion T is non-negative.
% If f/65536 > 1, the analysis is as above : some dimensions T sp
% are not attainable as D uu, but the formula
%     N=round((2T+1)*32768/f)
% will give a suitable decimal D via \the\dimexpr N sp\relax.
% (if T=0, we get N=0 as 32768/f<0.5)
% This D will let TeX convert D uu into T sp, if the dimension
% is attainable else it will be a closest match
% either from above or below (not necessarily nearest overall).
%
% If f/65536=1, attention that above formula would give N=1 for
% T=0 (was bug #4).
%
% If f/65536<1, all dimensions Tsp are attainable as D uu. Indeed
% D uu is parsed by TeX via N=round(D*65536), then T=trunc(N*phi),
% with phi=f/65536. Starting from T we need to find an N such that
% T/phi <= N< (T+1)/phi. 
%
% This is equivalent to ceil(T/phi)<= N < ceil((T+1)/phi)
%
% Now obsolete remark: let v=(T+0.5)/phi. As its
% distance to the extremities is 0.5/phi>0.5, (phi>1) its rounding M
% to an integer verifies automatically T/phi < M < (T+1)/phi, so
% is a valid candidate. This was used at 0.99 release.
% (it is funny that N=round((2T+1)*32768/f) works for all f>0
%  *except* f=65536).
%
% But the 0.999 release tries to use ceil(T/phi) formula rather
% as it is closer to naive expectation "dim1/dim2" of a division.
%
% It is not obvious to compute this ceil(T/phi) without overflow
% (and within the \numexpr tools where division rounds).
% See implementation.
%
% Implementation
% ==============
%
% \texdimenwithunit{dim1}{dim2}
%
% First done at 0.99, then refactored at 0.999:
% - to add support for dim2<0pt
% - to handle differently the dim2<1pt case and make the output
%   closer to mathematical dim1/dim2
%
% To handle dim2<0pt, we simply simultaneously do
% dim1<-- -dim1 and dim2<-- -dim2.
%
% dim2=0pt is not intercepted and will cause division by zero low-level error.
% Code comments below were not adjusted and handle only dim2>0pt.
%
% We first get f from dim2 and branch according to whether f>65536,
% or f<=65536.
% We will also need to check the sign of T (dim1=T sp).
% f>65536: we compute round((2T+1)*32768/f)
% f=65536: merged with f<65536 branch
% f<65536: we compute ceil(T*65536/f)
%
%          rationale: round((2T+1)*32768/f) which was used at 0.99
%          would be ok [if f<65536 not f=65536 ! cf #3, #4] 
%          also for this branch
%
%          BUT it diverges noticeably from naive expectation
%          dim1/dim2 "=" T*65536/f the more so when f is small.
%          See issue #16 and also the discussion at #13.
%
%          As was explained in comments to issue #2 which asked for a
%          \texdimenwithunit the ceil(T*65536/f) is the smallest
%          allowable choice
%
%          To avoid arithmetic overflow issues we first do the euclidean
%          division T = k f + r, 0<= r < f
%
%          The final result in sp units will be k*65536 + ceil(r * 65536/f)
%          We don't do this k*65536 explicitly as it may overflow
%          but output the decimal k + E where E is the conversion
%          to a decimal 0.ddddd or 1.0 of "ceil(r * 65536/f) sp"
%          (which is at most 65536sp=1pt so E is at most 1.0).
%
%          To compute the Euclidean quotient k in \numexpr we use there
%          (2T-f)/(2f) i.e. round((2T-f)/2f) = trunc(T/f)
%          as we are careful to never have T=0 inthere...
%
%          Computing C = ceil(r * 65536/f) in \numexpr is the delicate
%          part, as r can be as large as f-1 hence 65535 and r*65536 would
%          overflow. We could compute R=round(r*65536/f) ("scaling operation")
%          then C=R+1 if R*f-65536*r<0, else C=R.
%
%          The problem is then: how to get the sign of R*f-65536*r without
%          overflow? I considered various ways.
%
%          But in the end, opted for simply this:
%          - get R=round(r*65536/f) as \the\numexpr r*65536/f ("scaling" no overflow)
%          - hence get E pt=\the\dimexpr R sp
%          - let TeX compute E<f sp>: if it turns out < r sp,
%            then C=R+1,
%            else C=R. Done.
%
\def\texdimenwithunit#1#2{\expandafter\texdimenwithunit_i
% no premultiplication of dim1 by 2 as was done for technical
% reasons when dim2<1pt branch used round((2T+1)*32768/f)
    \the\numexpr\dimexpr#2\expandafter;\the\numexpr\dimexpr#1;}%
%
% so tempted to do \input xintkernel.sty to have some utilities...
% not even a \@gobble in Plain...
\let\texdimenorthat\texdimenfirstofone
\def\texdimendothis#1#2\texdimenorthat#3{\fi#1}%
\def\texdimengobtominus#1-{}%
\def\texdimenwithunit_i#1{%
     \texdimengobtominus#1\texdimenwithunit_switchsigns-%
     \texdimenwithunit_j#1%
}%
\def\texdimenwithunit_switchsigns-\texdimenwithunit_j-#1;#2%
{%
% due to \texdimenwithunit_Bneg we can not simply prefix dim1
% with -, as -0 is bad there. So let's check also if #2 is 0
    \texdimenzerominusfork
      #2-\texdimenwithunit_Bzero % abusive double usage
      0#2\texdimenwithunit_j     % abusive shortcut
       0-{\texdimenwithunit_ic#2}%
    \krof
    #1;%
}%
\def\texdimenwithunit_ic#1#2;{\texdimenwithunit_j#2;-#1}%
\def\texdimenwithunit_j#1;#2{%
        % \ifnum#1=\p@\texdimendothis\texdimenwithunit_p@\fi
        \ifnum#1>\p@\texdimendothis\texdimenwithunit_A\fi
        \texdimenorthat\texdimenwithunit_B#2#1;%
}%
% not needed, as the "ceil" branch is fine to use for f = 65536
% \def\texdimenwithunit_p@#1#2;#3;{%
%     \expandafter\texdimenstrippt\the\dimexpr#1#3sp/2\relax
% }%
% unit>1pt, handle this as for bp. Attention it would be wrong for
% unit=1pt!
\def\texdimenwithunit_A#1#2;#3;{%
    \expandafter\texdimenstrippt
    \the\dimexpr\numexpr(2*#1#3+\if-#1-\fi1)*32768/#2sp\relax
    % - fine if dim1>0, <0, or =0
    % - with *\p@ better but an early doubled dim2 would complicate 1pt
    % test and not sure if doing \p@/(2*#2) here advantageous
}%
% unit<=1pt.
% if dim1<0, simply negate result for dim1>0 as it can not possibly be 0.0
% Indeed T*65536/f will be at least 1 so its ceil also (in fact ceil
% will even be at least 2 if f<65536).
% The dim1=0 case must get filtered out due to way of calculating the
% "ceil" in \numexpr
\def\texdimenzerominusfork #10-#2#3\krof {#2}%
\def\texdimenwithunit_B#1{\texdimenzerominusfork
                           #1-\texdimenwithunit_Bzero
                           0#1\texdimenwithunit_Bneg
                           0-\texdimenwithunit_Ba
                          \krof#1}%
\def\texdimenwithunit_Ba#1#2;#3;{%
    % no overflow possible from 2*#1#3=2*dim1
    \expandafter\texdimenwithunit_Bb\the\numexpr(2*#1#3-#2)/(2*#2);#1#3;#2;%
}%
% now k;T;f;. Get the remainder r=T-k*f
\def\texdimenwithunit_Bb#1;#2;#3;{%
    \expandafter\texdimenwithunit_Bc\the\numexpr#2-#1*#3;#1;#3;%
}%
% now r;k;f;. We can start \the\numexpr k+ ....
% and there we will need to get R=round(r*65536/f), Ept=Rsp,
% check if E"f sp"<"r sp", if yes replace R by R+1 else keep E etc.
\def\texdimenwithunit_Bc#1;#2;#3;{%
% \the\numexpr k+0.ddddd is handy because it can well be actually
% \the\numexpr k+1.0, now that we use ceil approach in this branch
    \the\numexpr#2+\expandafter\texdimenwithunit_Bd
                   \the\numexpr #1*\p@/#3;#1;#3;%
}%
% R;r;f;
\def\texdimenwithunit_Bd#1;{%
    \expandafter\texdimenwithunit_Be\the\dimexpr#1sp;#1;%
}%
% Ept;R;r;f;
{\catcode`P 12\catcode`T 12
\lowercase{\gdef\texdimenwithunit_Be#1PT};#2;#3;#4;{%
    \ifdim#1\dimexpr#4sp<#3sp \texdimenwithunit_Bf#2\fi
% #1=0.ddd... or 1.0 but has no end marker so in _Bf we would
% not know how to get rid of it. Or I could have used dothis/orthat
    \texdimenfirstofone{#1}%
    }%
}%
% add a dimen \onesp?
\def\texdimenwithunit_Bf#1\fi\texdimenfirstofone#2{\fi
    \expandafter\texdimenstrippt\the\dimexpr#1sp+1sp\relax
}%
% Here definitely not caring about f-expandability. Or efficiency.
\def\texdimenwithunit_Bneg-{-\texdimenwithunit_Ba{}}%
\def\texdimenwithunit_Bzero#1;#2;{0.0}%
\texdimensendinput
%! Local variables:
%! mode: TeX
%! fill-column: 1000
%! sentence-end-double-space: t
%! End:
